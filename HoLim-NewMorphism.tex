%- HandOut Flag -----------------------------------------------------------------------------------------
\makeatletter
\@ifundefined{ifHandout}{%
  \expandafter\newif\csname ifHandout\endcsname
}{}
\makeatother

%- D0cum3nt ----------------------------------------------------------------------------------------------
\documentclass[beamer,10pt]{standalone}   
%\documentclass[beamer,10pt,handout]{standalone}  \Handouttrue  

\ifHandout
	\setbeameroption{show notes} %print notes   
\fi

	
%- Packages ----------------------------------------------------------------------------------------------
\usepackage{custom-style}
\usepackage{math}




%--Beamer Style-----------------------------------------------------------------------------------------------
\usetheme{toninus}
\usepackage{animate}
\usetikzlibrary{positioning, arrows}
\usetikzlibrary{shapes}

%===========================================================%
\begin{document}
%===========================================================%

%-----------------------------------------------------------%
\subsection{From twisted to untwisted higher Courant algebras}
%-----------------------------------------------------------%

%-----------------------------------------------------------%
\begin{frame}{From twisted to untwisted higher Courant algebras (1)}
	%
	\begin{block}{Goal:}
		Exhibit an $L_\infty$-morphism
		$$ \Courant[\sigma]{r-1} \to \Courant{r}$$
	\end{block}
	\vfill \pause	

	\begin{exblock}[Lower case: For $r=1,2$ we have:]
	
	\begin{displaymath}
		\begin{tikzcd}[ampersand replacement=\&,column sep = small]
			\Courant[\sigma]{0} \ar[r,phantom,":="] \ar[d]
			\&[1em]
			0 \ar[r] \ar[d]
			\& \Omega^0\oplus \X \ar[d,"{(\d,\id)}"]
			\\
			\Courant{1} \ar[r,phantom,":="] 
			\&
			\Omega^0 \ar[r,"{(\d,0)}"] 
			\& 
			\Omega^1\oplus \X
		\end{tikzcd}
	\end{displaymath}
	\cite{Zambon2012} proved that  the vertical arrows are the linear part of a $L_\infty$-morphism $\Courant[\sigma]{0} \to \Courant[\sigma]{1}$.
	\end{exblock}
	\vfill
	
\end{frame}
\note[itemize]
{
	\item 
}

%-----------------------------------------------------------%
\begin{frame}{From twisted to untwisted higher Courant algebras (2)}
	%
	More generally we have a morphism of chain complexes:
	\begin{displaymath}
		\begin{tikzcd}[ampersand replacement=\&,column sep = small]
			\Courant[\sigma]{r-1} %\ar[r,phantom,":="] \ar[d]
			\&[1em]
			0 \ar[r] \ar[d]
			\& \Omega^0 \ar[d,"\d"] \ar[r]
			\& \Omega^1 \ar[d,"\d"] \ar[r]
			\& \cdots \ar[r]
			\& \Omega^{r-2} \ar[d,"\d"] \ar[r,"{(\d,0)}"]
			\&[.5em]  \Omega^{r-1} \oplus \X \ar[d,"{(\d,\id)}"]
			\\
			\Courant{r} %\ar[r,phantom,":="]
			\& \Omega^0 \ar[r]
			\& \Omega^1 \ar[r]
			\& \Omega^2 \ar[r]
			\& \cdots \ar[r]
			\& \Omega^{r-1} \ar[r,"{(\d,0)}"]
			\&\Omega^{r} \oplus \X
		\end{tikzcd}
	\end{displaymath}

	\vfill\pause

	\begin{thmblock}[{[Fiorenza-M. 2025]}]
		The above chain complex morphism is the linear part of a $L_\infty$-morphism $$\Courant[\sigma]{r} \to \Courant{r+1}~.$$
	\end{thmblock}

	\vfill
\end{frame}
%-----------------------------------------------------------%

%-----------------------------------------------------------%
\begin{frame}{Idea of the proof:}
	%
	Recall that $$\Courant[\sigma]{r-1} = \hofib(\mathcal{C}_{r,\sigma}^{\geq 0} \hookrightarrow \mathcal{C}_{r,\sigma}) $$
	\pause
	\vfill

	We want to exhibit a morphism:
		\begin{displaymath}
			\begin{tikzcd}[ampersand replacement=\&]
				\hofib(\mathcal{C}_{r,\sigma}^{\geq 0} \hookrightarrow \mathcal{C}_{r,\sigma}) \ar[d]
				\\
				\hofib(\mathcal{C}_{r+1}^{\geq 0} \hookrightarrow \mathcal{C}_{r+1})
			\end{tikzcd}
		\end{displaymath}
	\pause
	\vfill

	This should be naturally\footnote{By the universal property of homotopy limits.} induced by a (homotopy) commutative diagram of dg-Lie algebras (or $L\infty$-algebras):
		\begin{displaymath}
			\begin{tikzcd}[ampersand replacement=\&]
				\mathcal{C}_{r,\sigma}^{\geq 0} \ar[r,hook] \ar[d ]
				\& \mathcal{C}_{r,\sigma} \ar[d, ]
				\\
				\mathcal{C}_{r+1}^{\geq 0} \ar[r,hook] \& \mathcal{C}_{r+1} 
			\end{tikzcd}
		\end{displaymath}
	
\end{frame}
\note[itemize]
{
	\item 
	\item 
}
%-----------------------------------------------------------%

%-----------------------------------------------------------%
\begin{frame}{A technical lemma (1)}
  Inside $\mathcal{C}_r$ there is a smaller, mora manageble, dg-Lie subalgebra with the same negative part.
  \vfill \pause

  \begin{defblock}[Cartan dg-Lie algebra]
  	Is a $2$-terms dg-Lie algebra $\Cartan \subseteq \Der(\Omega)$ given by:
  \begin{displaymath}
	\begin{tikzcd}[ampersand replacement=\&,row sep = small]
		\text{\tiny (deg $-1$)} \& \text{\tiny (deg $0$)}\&[2em]
		\\[-.7em]
		\iota_{\X} \ar[r,"\delta"] \& \Lie_{\X} \ar[r,phantom,";"] \& \delta:= [\d, \cdot]
	\end{tikzcd}
  \end{displaymath}
  \end{defblock}
  \vfill \pause

  $\Omega$ is a module for $\Der(\Omega)$ and so it is a module for $\Cartan$.
  \\
  $\Rightarrow$\quad By shifting degrees, $\Omega[r]$ is a module for $\Cartan$.
  \vfill 

\end{frame}
\note[itemize]
{
	\item 
	\item 
}
%-----------------------------------------------------------%

%-----------------------------------------------------------%
\begin{frame}{A technical lemma (2)}
	  \begin{defblock}[The \emph{small} dgla]
		We can form the dg-Lie algebra
		$$ \g_r := \Cartan \ltimes \Omega[r]$$
		\pause
		\smallskip

		Observe that any $\sigma \in \Omega^{r+1}_{\mathrm{cl}})$ is a MC (degree 1) element of $\g_r$.
		\pause
		\smallskip
		
		We can consider the twisted dg-Lie algebra $(\g_{r,\sigma}, \d_\sigma, \lbrace \cdot,\cdot\rbrace_\sigma)$ with:
	\begin{displaymath}
		\d_\sigma (\omega) = \d \omega~, \qquad
		\d_\sigma (\iota_X) = \Lie_X + \iota_X \sigma~, \qquad
		\d_\sigma(\Lie_X) = -\Lie_X\sigma
	\end{displaymath}
	\end{defblock}
	\vfill\onslide<4->{

	\begin{lemblock}[{$\hofib(\mathcal{C}_{r,\sigma}^{\geq 0} \hookrightarrow \mathcal{C}_{r,\sigma}) \cong \hofib(\g_{r,\sigma}^{~\geq 0} \hookrightarrow \g_{r,\sigma})$}]

			We have\footnote{the left square is in DGLAs, the right one is in dg vectors spaces}:
			\vspace{-1em}
	\begin{displaymath}
		\begin{tikzcd}[ampersand replacement=\&]
			\g_{r,\sigma}^{~\geq 0} \ar[r,hook] \ar[d,hook] \&
			\g_{r,\sigma} \ar[d,hook] \ar[r,two heads] \&[1em]
			\g_{r,\sigma}^{~< 0} \ar[d,equal] \\
			\mathcal{C}_{r,\sigma}^{~\geq 0} \ar[r,hook] \&
			\mathcal{C}_{r,\sigma} \ar[r,two heads] \&
			\mathcal{C}_{r,\sigma}^{~< 0}
		\end{tikzcd}
	\end{displaymath}

	Therefore $\mathcal{C}_{r,\sigma}^{~< 0}[-1] = \g_{r,\sigma}^{~<0}[-1]$ as $L_\infty$-algebras with the \cite{Getzler1991} derived brackets.
	\end{lemblock}
	}

\end{frame}
%-----------------------------------------------------------%

%-----------------------------------------------------------%
\begin{frame}
  So we can replace $\mathcal{C}_{r,\sigma}, \mathcal{C}_{r+1}$ with $\g_{r,\sigma}, \g_{r+1}$ in the previous commuative square:
	\begin{displaymath}
		\begin{tikzcd}[ampersand replacement=\&]
			\g_{r,\sigma}^{~\geq 0} \ar[r,hook] \ar[d, ]
			\& \g_{r,\sigma} \ar[d, ]
			\\
			\g_{r+1}^{~\geq 0} \ar[r,hook] \& \g_{r+1}
		\end{tikzcd}
	\end{displaymath}
	\vfill

	Moreover, if this has to be true for any $\sigma \in \Omega^{r+1}_{\mathrm{cl}}(M)$, it has to be true for $\sigma = 0$. So let begin with the simpler situation
	\begin{displaymath}
		\begin{tikzcd}[ampersand replacement=\&]
			\g_{r}^{~\geq 0} \ar[r,hook] \ar[d, ]
			\& \g_{r} \ar[d, ]
			\\
			\g_{r+1}^{~\geq 0} \ar[r,hook] \& \g_{r+1}
		\end{tikzcd}
	\end{displaymath}
	\vfill
	Let us consider the right-hand vertical arrow:
\end{frame}
%-----------------------------------------------------------%

%-----------------------------------------------------------%
\begin{frame}
		The vertical arrows forms a morphism of cochain complexes $\Phi_1:\g_{r,\sigma}\to \g_{r+1}$:
	\begin{displaymath}
		\begin{tikzcd}[ampersand replacement=\&, column sep = small]
			0\ar[r]\ar[d] \&
			\Omega^0 \ar[r] \ar[d,"\d"] \&
			\Omega^1 \ar[r] \ar[d,"\d"] \&
			\Omega^2 \ar[r] \&
			\cdots \ar[r] \&
			\Omega^{r-1}\oplus \iota_{\X} \ar[d,"\d\oplus \id"] \ar[r] \&
			\Omega^{r}\oplus \Lie_{\X} \ar[r]\ar[d,"\d\oplus\id"] \&
			\Omega^{r+1} \ar[d,"\d"] \ar[r,phantom,"\cdots"] \& \phantom{.}
			\\
			\Omega^{0} \ar[r] \&
			\Omega^{1} \ar[r] \&
			\Omega^{2} \ar[r] \&
			\cdots \& \cdots \ar[r] \&
			\Omega^{r}\oplus\iota_{\X} \ar[r] \&
			\Omega^{r+1}\oplus \Lie_{\X} \ar[r] \&
			\Omega^{r+2} \ar[r,phantom,"\cdots"]  \& \phantom{.}
		\end{tikzcd}
	\end{displaymath}
	%
	\vfill

	Is $\Phi_1$ a dg-Lie algebra morphism?
	$$ \Phi_1\big(\lbrace a, b \rbrace\big) \overset{?}{=} \big\lbrace \Phi_1(a), \Phi_1(b) \big\rbrace$$
	\vfill

	No: the de Rham differential does not commute with the contractions:
	$$ \Phi_1\big(\lbrace \iota_X, \omega \rbrace\big) \neq \big\lbrace \Phi_1(\iota_X), \Phi_1(\omega)\big\rbrace~.$$
\end{frame}
%-----------------------------------------------------------%

%-----------------------------------------------------------%
\begin{frame}
	Let us try and cure this by setting
	\begin{align*}
		\Phi_2(\iota_X,\omega) &=~ \iota_X \omega \\
		\Phi_2(\omega,\iota_X) &=~ \pm \iota_X \omega \\
		\Phi_2(a,b) &=~ 0 \qquad \text{in all other cases}
	\end{align*}
	\vfill

	$\Phi= (\Phi_1,\Phi_2,0,\dots)$ is a $L_\infty$-morphism $\g_r \to \g_{r+1}$!
	\vfill

	Does it maps $\g_{r,\sigma}^{~\geq 0}$ to $\g_{r+1}^{~\geq 0}$? Yes!\footnote{ We need to check that $\Phi_2(a,b)\in \g_{r+1}^{~\geq 0}$ when $|a|=|b|=0$. this is true since for these elements $\Phi_2(a,b)=0$.}
	\vfill

	So we have a strictly commutative diagram of $L_\infty$-morphisms:
	\begin{displaymath}
		\begin{tikzcd}[ampersand replacement=\&]
			\g_{r,\sigma}^{~\geq 0} \ar[r,hook] \ar[d, "\Phi \vert_{\geq 0}"']
			\& \g_{r,\sigma} \ar[d, "\Phi"]
			\\
			\g_{r+1}^{~\geq 0} \ar[r,hook] \& \g_{r+1}
		\end{tikzcd}
	\end{displaymath}
\end{frame}
%-----------------------------------------------------------%

%-----------------------------------------------------------%
\begin{frame}
  What for arbitrary $\sigma \in \Omega^{r+1}_{\mathrm{cl}}(M)$?
  We use $\sigma$ to twist everything:
	\begin{displaymath}
		\begin{tikzcd}[ampersand replacement=\&]
			\g_{r,\sigma}^{~\geq 0} \ar[r,hook] \ar[d, "\Phi_\sigma \vert_{\geq 0}"']
			\& \g_{r,\sigma} \ar[d, "\Phi_\sigma"]
			\\
			\g_{r+1,\Phi(\sigma)}^{~\geq 0} \ar[r,hook] \& \g_{r+1,\Phi(\sigma)}
		\end{tikzcd}
	\end{displaymath}
	\vfill
	Where:
	\begin{align*}
		\big(\Phi_\sigma \big)_j (\cdots) 
		&=~
		\sum_{k=0}^\infty \frac{\Phi_{k+j}}{k!} (\overbrace{\sigma,\cdots,\sigma}^{\text{\tiny $k$ copies}},\cdots)
		\\
		\Phi(\sigma) &=~ \sum_{k=0}^\infty \frac{\Phi_k(\sigma,\cdots,\sigma)}{k!} = \d(\sigma) + \frac{1}{2}\cdot 0 = 0
	\end{align*}
\end{frame}
%-----------------------------------------------------------%

%-----------------------------------------------------------%
\begin{frame}
	So
	\begin{displaymath}
		\begin{tikzcd}[ampersand replacement=\&]
			\g_{r,\sigma}^{~\geq 0} \ar[r,hook] \ar[d, "\Phi_\sigma \vert_{\geq 0}"']
			\& \g_{r,\sigma} \ar[d, "\Phi_\sigma"]
			\\
			\g_{r+1 }^{~\geq 0} \ar[r,hook] \& \g_{r+1 }
		\end{tikzcd}
	\end{displaymath}
	\vfill
	\begin{block}{Details}
		\begin{itemize}
			\item we have to check that $\Phi_\sigma$ preseves the positive part. This is immediate since $(\Phi_\sigma)_2 = \Phi_2$.
			\item Moreover, we have 
			$$ (\Phi_\sigma)_1 \big\vert_{\geq 0} = \Phi_1 \big\vert_{\geq 0} + \frac{1}{2}\cancel{\Phi_2(\sigma,\cdot)\big\vert_{\geq 0}} = \Phi_1\big\vert_{\geq 0} $$
		\end{itemize}
	\end{block}
	\vfill
	
\end{frame}
%-----------------------------------------------------------%


%-----------------------------------------------------------%
\subsection{From Rogers to Courant algebras}
%-----------------------------------------------------------%

%-----------------------------------------------------------%
\begin{frame}
	Consider $\mathcal{C}_r^{\geq 0}[-1] \hookrightarrow \mathcal{C}_r[-1]$
	\vfill

	Observe that $\mathcal{C}_r^{<0} = C^\infty(T^*[r]T[1]M)^{<r}=$
	\begin{displaymath}
		\begin{tikzcd}[ampersand replacement=\&]
			\Omega^0 \ar[r,"\d"]\&
			\Omega^1 \ar[r,"\d"]\&
			\Omega^2 \ar[r,"\d"]\&
			\cdots \ar[r,"\d"]\&
			\Omega^{r-2} \ar[r,"{(\d,0)}"]\&
			\Omega^{r-1}\ar[d,phantom,"\oplus"]\\[-.7em]
			\&\&\&\&\&\X
 		\end{tikzcd}
	\end{displaymath}
	\vfill

	\begin{displaymath}
		\begin{tikzcd}[ampersand replacement=\&,column sep = small]
			\& \text{\small (deg $1-r$)} \& \cdots \& \&\text{\small (deg $-1$)} \& \text{\small (deg $0$)} \\[-.5em]
			\Rogers[\sigma]{r-1}   \ar[r,phantom,":="]
			\&
			\Omega^0 \ar[r,"{\d}"] \ar[d,"id"]
			\&
			\Omega^1 \ar[r,"{\d}"] \ar[d,"id"]
			\&
			\cdots
			\&
			\Omega^{r-2} \ar[r,"{(0,\d)}"] \ar[d,"id"]
			\& \Ham_{\sigma}^{(r-1)} \ar[d,hook]
			\\
			\Courant[\sigma]{r-1} \ar[r,phantom,":="]
			\&
			\Omega^0 \ar[r,"{\d}"]
			\&
			\Omega^1 \ar[r,"{\d}"] 
			\&
			\cdots
			\&
			\Omega^{r-2} \ar[r,"{(0,\d)}"]
			\& \Gamma(E^{(1)})
		\end{tikzcd}
	\end{displaymath}
	\vfill

	\cite{Miti2021} proved that the vertical arrows are the linear part of a $L_\infty$-morphism $\Rogers[\sigma]{r-1} \to \Courant[\sigma]{r-1}$.
\end{frame}
%-----------------------------------------------------------%



%-----------------------------------------------------------%
\begin{frame}{The \cite{Miti2021} $L_\infty$-morphism revisited}
	We have the following:
	\begin{displaymath}
		\begin{tikzcd}[ampersand replacement=\&]
			\Rogers[\sigma]{r-1} \ar[d] \ar[r] \&
			\X_{\ham,\sigma} \ar[r,"\iota_{\dots}\sigma"] \ar[d,"\Lie"]\&
			(\Omega^0\to\Omega^1\to\cdots\to\Omega^{r-1}\to \d \Omega^{r-1}) \ar[d,hook] 
			\\
			\Courant[\sigma]{r-1} \ar[r] \&
			\g_{r+1,\sigma}^{~\geq 0} \ar[r,hook] \ar[ur,Rightarrow]\& 
			\g_{r+1,\sigma}
		\end{tikzcd}
	\end{displaymath}
	\vfill

	Since \cite{Fiorenza2014a} shows that:
	\begin{itemize}
		\item[$\bullet$] $\iota_{\ldots}\sigma$ (multicontractions with $\sigma$) is a $L_\infty$-morphism
		\item[$\bullet$] $\Rogers[\sigma]{r-1} =\hofib(\iota_{\ldots}\sigma)$\footnote{Is it better to say that is a model of the homotopy fiber?}
	\end{itemize}
	\vfill

	The homotopy is given by $$ e^{\iota} * \Lie = \iota_{\ldots}\sigma$$
	which is basically Cartan's magic formula for multivector fields\footnote{Twisting the differential, i.e. changing $\d$ in $\d_\sigma$, the multicontractions $\iota_{x_1\wedge\cdots\wedge x_k}\sigma$ appear.}
	$$[\d, \iota_{x_1\wedge\cdots\wedge x_k}] = \Lie_{x_1\wedge\cdots\wedge x_k}$$

\end{frame}
%-----------------------------------------------------------%



%-----------------------------------------------------------%
\begin{frame}{Conclusion:}

	\vfill

	\begin{displaymath}
		\begin{tikzcd}[ampersand replacement=\&]
			\Rogers[\sigma]{r-1} \ar[d] \ar[r] \&
			\X_{\ham,\sigma} \ar[r,"\iota_{\dots}\sigma"] \ar[d,"\Lie"]\&
			(\Omega^0\to\Omega^1\to\cdots\to\Omega^{r-1}\to \d \Omega^{r-1}) \ar[d,hook] 
			\\
			\Courant[\sigma]{r-1} \ar[r] \ar[d]\&
			\g_{r+1,\sigma}^{~\geq 0} \ar[r,hook] \ar[ur,Rightarrow] \ar[d,"\Phi\big\vert_{\geq 0}"]\& 
			\g_{r+1,\sigma} \ar[d,"\Phi"]
			\\
			\Courant{r} \ar[r] \&
			\g_{r}^{\geq 0} \ar[r,hook]\&
			\g_{r}
		\end{tikzcd}
	\end{displaymath}
\end{frame}
%-----------------------------------------------------------%



%-----------------------------------------------------------%
\ifstandalone
% https://en.wikibooks.org/wiki/LaTeX/Bibliographies_with_biblatex_and_biber
\begin{frame}[t,allowframebreaks]{Partial Bibliography}
	\nocite{Miti2021}
	\bibliographystyle{alpha}
	\bibliography{bibfile}
\end{frame}
\fi
%-----------------------------------------------------------%

\end{document}