%+----------------------------------------------------------------------------+
%| SLIDES: main file
%| Contents:	- 60 minutes (extimated duration 3 minutes per slide )
%|				- 10 slides  Introduction and Background
%|				- 10 slides  Results
%| Author: Antonio miti
%| Place: 
%| Date: 
%+----------------------------------------------------------------------------+


%- HandOut Flag -----------------------------------------------------------------------------------------
	\newif\ifHandout
	%\Handouttrue  %uncomment for the printable version
	%Handling of flags it is not preserved when passing to standalone-subfiles!


%- D0cum3nt ----------------------------------------------------------------------------------------------
\ifHandout
	\documentclass[handout,10pt]{beamer}   
	\setbeameroption{show notes} %print notes   
\else
	\documentclass[10pt]{beamer}
\fi




%- Packages ----------------------------------------------------------------------------------------------
\usepackage{custom-style}
\usepackage{math}


%--Beamer Style-----------------------------------------------------------------------------------------------
\usetheme{toninus}


\usetikzlibrary{backgrounds}
  \tikzset{
    invisible/.style={opacity=0},
    visible on/.style={alt=#1{}{invisible}},
    alt/.code args={<#1>#2#3}{%
      \alt<#1>{\pgfkeysalso{#2}}{\pgfkeysalso{#3}} % \pgfkeysalso doesn't change the path
    },
  }



%- T1tle P4g3 -------------------------------------------------------------------------------------------
\title{A canonical morphism between \\ twisted and untwisted higher Courant \texorpdfstring{$L_\infty$}{Lie infinity} algebras} 
\subtitle{}
\author[AMM]{\href{www.antoniomiti.it}{\textbf{Antonio Michele MITI}} \\ \small(j.w. Domenico FIORENZA)}
%\institute[UCSC and KU Leuven]{
%  \begin{tabular}[h]{ccc}
%      Università Cattolica del Sacro Cuore & $\qquad$ & KU Leuven \\
%      Brescia, Italy & & Leuven, Belgium \\
%      \href{https://dipartimenti.unicatt.it/dmf-home?rdeLocaleAttr=it}{\includegraphics[width=3.5cm]{Logos/UnicattBS-logo}} & & 
%      \href{https://wis.kuleuven.be/english}{\includegraphics[width=4cm]{Logos/KULeuven_logo}}
%  \end{tabular}      
%}
%\institute[Mpim]{
%	MPIM, Bonn, Germany 
%	\\
%	\vspace{.5em}
%	\href{https://www.mpim-bonn.mpg.de/}{\includegraphics[width=6cm]{./Logos/Mpim_logo}}
%}
\institute[SUR]{
	Sapienza Università di Roma \\
	Rome, Italy 	\\
	\vspace{.5em}
  \begin{tabular}[h]{ccc}
      \href{https://dipartimenti.unicatt.it/dmf-home?rdeLocaleAttr=it}{\includegraphics[width=6cm]{./Logos/Sur_logo}} & & 
      \href{https://wis.kuleuven.be/english}{\includegraphics[width=3cm]{Logos/Civis_logo}}
  \end{tabular}    
}
\date[Template_21] % (optional, should be abbreviation of conference name)
{	
	{\vskip 1ex}
	\href{https://www.mat.uniroma2.it/~kowalzig/ws.html}{Geometry \& Topology in Rome}
	\\
	July 4, 2025
}



 



%---------------------------------------------------------------------------------------------------------------------------------------------------
%- D0cum3nt ----------------------------------------------------------------------------------------------------------------------------------
\begin{document}
%-------------------------------------------------------------------------------------------------------------------------------------------------

\begin{frame}  % Alternative: \maketitle outside of frame
	\titlepage
	\ifHandout
		\tikz[overlay,remember picture]
		{
	    	%	\node at ($(current page.west)+(1.5,0)$) [rotate=90] {\Huge\textcolor{gray}{\today}};
	    	\node[        
	    		draw,
	    		shape border rotate=90,
			isosceles triangle,
			isosceles triangle apex angle=90,
			fill=yellow]
	        		at ($(current page.north east)-(1,1)$) [rotate=-45] {\textcolor{red}{Handout version}};
		}
	\fi
	% European Commission	
		\tikz[overlay,remember picture]
		{
	    	%	\node at ($(current page.west)+(1.5,0)$) [rotate=90] {\Huge\textcolor{gray}{\today}};
	    	\node[        
	    		text width = 3.5cm
			]
	        		at ($(current page.south west)+(2.0,1.0)$) [rotate=0] {

					\href{https://dipartimenti.unicatt.it/dmf-home?rdeLocaleAttr=it}{\includegraphics[width=\textwidth]{./Logos/Eu-h2020}}
 		
	        		};
		}
	\end{frame}
	\addtocounter{framenumber}{-1}
\note{
	%\textbf{\underline{OUTLINE}}:
	%\tableofcontents
	\textbf{Abstract:}
	\\
In "Lie infinity algebras and higher analogues of Dirac structures and Courant algebroids" \cite{Zambon2012}, Marco Zambon constructs a Lie infinity algebra associated with any higher standard Courant algebroid (also known as a Vinogradov algebroid), and exhibits an explicit Lie infinity morphism from the Lie algebra associated with a standard Lie algebroid twisted by a closed 2-form to the Lie-2 algebra of the standard Courant algebroid. He poses the question of whether analogous canonical morphisms exist in higher degrees—namely, for any standard higher Courant algebroids twisted by closed $(n+1)$-forms.
In this talk, we present a general framework that naturally yields such canonical Lie infinity morphisms for arbitrary n, clarifying the geometric and homotopical structures underlying these constructions.
Time permitting, we also discuss how this framework accommodates the canonical morphism between the observable Lie infinity algebra of a pre-n-plectic manifold and the higher Courant algebra we described in "Observables on multisymplectic manifolds and higher Courant algebroids" \cite{Miti2024}.
This is joint work with Domenico Fiorenza.
    
}
%---------------------------------------------------------------------------------------------------------------------------------------------------





%-----------------------------------------------------------%
\section{Introduction}
%-----------------------------------------------------------%
	%- HandOut Flag -----------------------------------------------------------------------------------------
\makeatletter
\@ifundefined{ifHandout}{%
  \expandafter\newif\csname ifHandout\endcsname
}{}
\makeatother

%- D0cum3nt ----------------------------------------------------------------------------------------------
\documentclass[beamer,10pt]{standalone}   
%\documentclass[beamer,10pt,handout]{standalone}  \Handouttrue  

\ifHandout
	\setbeameroption{show notes} %print notes   
\fi

	
%- Packages ----------------------------------------------------------------------------------------------
\usepackage{custom-style}
\usepackage{math}




%--Beamer Style-----------------------------------------------------------------------------------------------
\usetheme{toninus}
\usepackage{animate}
\usetikzlibrary{positioning, arrows}
\usetikzlibrary{shapes}

%===========================================================%
\begin{document}
%===========================================================%


%-----------------------------------------------------------%
%-------------------------------------------------------------------------------------------------------------------------------------------------
\begin{frame}[fragile]{Unpacking the title}
\tikzstyle{every picture}+=[remember picture]
	\begin{columns}
    	\begin{column}{.45\textwidth}
    		\onslide<5->{
			\tikz[baseline]{
		            \node[draw=orange!40,anchor=base,text width=5cm] (s1)
		            {\textbf{Goal:}
					is there a morphism 
					\\ \emph{from:} the $L_\infty$-algebra associated with $E^{(n)}$ and the twist $\omega$,
					\\ \emph{to:} the $L_\infty$-algebra associated with $E^{(n+1)}$ without twist?};
			}
		}
	\end{column}
    	\begin{column}{.45\textwidth}
    		\onslide<2->{
			 \tikz[baseline]{
		            \node[draw=blue!40,anchor=base, text width=5cm] (s2)
		            {Higher generalization of Lie algebras where the Jacobi identity is allowed only "up to homotopies".};
			}
		}
	\end{column}
	\end{columns}

	\vfill
	%A canonical morphism between \\ twisted and untwisted higher Courant $L_\infty$-algebras
	\begin{center}
		\large
		A
		 \tikz[baseline]{
		            \node[fill=orange!20,anchor=base] (t1)
		            {Canonical morphism};
			}
		between \\
		 \tikz[baseline]{
		            \node[fill=green!20,anchor=base] (t3)
		            {twisted};
		        } 
		and untwisted 
		 \tikz[baseline]{
	            \node[fill=red!20,anchor=base] (t4)
	            {higher Courant};
		}
		 \tikz[baseline]{
	            \node[fill=blue!20,anchor=base] (t2)
	            {$L_\infty$-algebras};
		}		
	\end{center}

	\vfill

	\begin{columns}
    	\begin{column}{.45\textwidth}
    		\onslide<4->{
	 		 \tikz[baseline]{
	            \node[draw=green!40,anchor=base,text width=5cm] (s3)
	            {any closed differential form $$\omega\in\Omega^{n+2}(M)$$ can be use to \emph{twist} the binary bracket on $\Gamma(E^{(n)})$.};
	         }
		}		   	
		\end{column}
    	\begin{column}{.45\textwidth}
    		\onslide<3->{
				\tikz[baseline]{
	            \node[draw=red!40,anchor=base,text width=5cm] (s4)
	            {\textbf{Higher Courant algebroids}\\
					smooth vector bundles
				$$ E^{(n)}= TM \oplus \wedge^{n}T^\ast M \to M$$

				 	the space of global sections $\Gamma(E^{(n)})$ forms an $L_\infty$-algebra!
				};
	           }	
			}
		\end{column}
	\end{columns}

	\begin{tikzpicture}[overlay]
        \path[->,draw=orange!40, line width=.5mm]<5-> (s1) edge [bend right] (t1);
        \path[->,draw=blue!40, line width=.5mm]<2-> (s2) edge [bend left] (t2);
        \path[->,draw=green!40, line width=.5mm]<4-> (s3) edge [bend left] (t3);
        \path[->,draw=red!40, line width=.5mm]<3-> (s4) edge [bend right] (t4);
	\end{tikzpicture}


\end{frame}
\note[itemize]{
	\item we deal with the category of \emph{L_\infty-algebras}. These are a generalziation of dg-Lie algebras, where we ask the Jacobi identity to be satisfied up to exact terms. The corresponding primitives are given by certain higher brackets.
	\item the (standard) higher Courant algebroid is part of a family of vector bundle over $M$. In degree 0 we have the standard Lie algebroid, in degree 1 the standard Courant algebroid an we can go higher in this sequence.
	\item 
	\item Conventions:
	\\- $M$ and $G$ are connected,
	\\- actions $\theta:G \curvearrowright M$ are always smooth
	\\- $\xi,\eta\in\mathfrak{g}$,
	\\- for $\mu\in\Omega^*(M,\mathfrak{g}^*)$ and $\xi\in\mathfrak{g}$, write
			\[
				\mu_\xi := \langle\mu,\xi\rangle \;{\color{black!50}\in\Omega^*(M)}
			\]
			for the ``$\xi$th component'' of $\mu$.
}
%-----------------------------------------------------------%

%-----------------------------------------------------------%
\outline
%-----------------------------------------------------------%

%-----------------------------------------------------------%
\subsection{Symplectic Case}
%-----------------------------------------------------------%
\begin{frame}{Symplectic Case}
	\begin{itemize}
		\item \textbf{Goal:} Find a canonical morphism between the twisted and untwisted higher Courant $L_\infty$-algebras.
		\item \textbf{Strategy:} Use the cone construction to obtain a homotopy fiber of the morphism.
		\item \textbf{Result:} The morphism is given by a certain 2-plectic structure on the cone.
	\end{itemize}
\end{frame}
\note[itemize]{
	\item 
}%-----------------------------------------------------------%
\subsection{2-plectic Case}
%-----------------------------------------------------------%
\begin{frame}{2-plectic Case}
	\begin{itemize}
		\item \textbf{Goal:} Extend the result to the 2-plectic case.
		\item \textbf{Strategy:} Use the cone construction to obtain a homotopy fiber of the morphism.
		\item \textbf{Result:} The morphism is given by a certain 3-plectic structure on the cone.
	\end{itemize}
\end{frame}
\note[itemize]{
	\item 
}
%-----------------------------------------------------------%
\subsection{Goals}
%-----------------------------------------------------------%
\begin{frame}{Goals}
	\begin{itemize}
		\item \textbf{Goal:} Find a canonical morphism between the twisted and untwisted higher Courant $L_\infty$-algebras.
		\item \textbf{Strategy:} Use the cone construction to obtain a homotopy fiber of the morphism.
		\item \textbf{Result:} The morphism is given by a certain $(n+1)$-plectic structure on the cone.
	\end{itemize}
\end{frame}
\note[itemize]{
	\item 
}



%-----------------------------------------------------------%
\ifstandalone
% https://en.wikibooks.org/wiki/LaTeX/Bibliographies_with_biblatex_and_biber
\begin{frame}[t,allowframebreaks]{Partial Bibliography}
	\nocite{Miti2021}
	\bibliographystyle{alpha}
	\bibliography{bibfile}
\end{frame}
\fi
%-----------------------------------------------------------%

\end{document}


%-----------------------------------------------------------%
\section{Background}
%-----------------------------------------------------------%
	\checkpoint	
	%- HandOut Flag -----------------------------------------------------------------------------------------
\makeatletter
\@ifundefined{ifHandout}{%
  \expandafter\newif\csname ifHandout\endcsname
}{}
\makeatother

%- D0cum3nt ----------------------------------------------------------------------------------------------
\documentclass[beamer,10pt]{standalone}   
%\documentclass[beamer,10pt,handout]{standalone}  \Handouttrue  

\ifHandout
	\setbeameroption{show notes} %print notes   
\fi

	
%- Packages ----------------------------------------------------------------------------------------------
\usepackage{custom-style}
\usepackage{math}




%--Beamer Style-----------------------------------------------------------------------------------------------
\usetheme{toninus}
\usepackage{animate}
\usetikzlibrary{positioning, arrows}
\usetikzlibrary{shapes}

%===========================================================%
\begin{document}
%===========================================================%




%-----------------------------------------------------------%
\subsection{Courant Algebroids}
%-----------------------------------------------------------%
%-----------------------------------------------------------%
\begin{frame}{Courant Algebroids \& generalized Geometry}


\end{frame}
\note[itemize]{
	\item   
}
%-----------------------------------------------------------%

%-----------------------------------------------------------%
\begin{frame}{Higher Courant Algebroids}
	\begin{defblock}[Higher Courant (Vinogradov) Algebroids]
		\includestandalone[width=0.95\textwidth]{Pictures/Figure_vinogradov}	
	\end{defblock}

	\begin{itemize}
		\item $(n=1) ~ \Rightarrow$ standard twisted \emph{Lie algebroid};
		\item $(n=2) ~ \Rightarrow$ standard twisted \emph{Courant algebroid};
	\end{itemize}

\end{frame}
\note[itemize]{
\item Att, questa definizione sarebbe lo standard twisted. E' possibile trovare in letteratura definizioni piu' generali corrispondenti alla nozione di Courant algebroid in astratto.
}
%-----------------------------------------------------------%

%-----------------------------------------------------------%
\begin{frame}{Vinogradov $L_\infty$-algebra}
	Vin. alg.oids are $NQ$-manifolds ($L_\infty$-algebroids).
	$\quad\Rightarrow\quad$ 
	Associated $L_\infty$-algebra.

	\begin{defblock}[Vinogradov $L_\infty$-algebra \cite{Zambon2012}]
		\includestandalone[width=0.95\textwidth]{Pictures/Figure_vinogradov-Linfty}	
	\end{defblock}
	\vfill
	\only<2->{
		\tikz[overlay,remember picture]
		{
			\node[rounded corners,
                 fill=gray!1,draw=gray!30,anchor=base]            
            	 (base) at ($(current page.south)+(0,.5)$) [rotate=-0,text width=10cm,align=center] { \footnotesize{\color{gray}{
            	 $e_i = \pair{X_i}{\alpha_i} \in \mathfrak{X}(M)\oplus \Omega^{n-1}(M)$ 
            	 \quad~,\qquad
            	 $f_i \in \bigoplus_{k=0}^{n-2}\Omega^k(M)$.
            	 }}};
		}			
	}

\end{frame}
\note[itemize]{
	\item The actions of non vanishing multi-brackets (up to permutations of the entries) on arbitrary vectors
are given in the slide.
	\item $\mu_2 \left(e_1,e_2\right) 	= [e_1,e_2]_\omega 
			= \pair{[X_1,X_2]}{\dd \langle e_1, e_2\rangle_- 
			+ (\iota_{X_1}\dd\alpha_2 - \iota_{X_2}\dd\alpha_1 + \iota_{X_1}\iota_{X_2}\omega)}$
	\item $				\mu_2 \left(e_1,f_2\right) = -\mu_2(f_2,e_1) = 
				\frac{1}{2} \mathcal{L}_{X_1} f_2 = \langle e_1, \dd f_2 \rangle_-$
	\item $k$-ary bracket for $k \ge 3$ an \emph{odd} integer:	 
		\begin{equation}\label{eq:VinoMultibrakAllaZambon_1}
			\begin{split}
				\mu_k(\varv_0,\cdots,\varv_{k-1})
				=&
				\left(\sum_{i=0}^{k-1} {(-)^{i-1}\mu_k(f_i+\alpha_i,X_0,\dots,\widehat{X_i},\dots,X_{k-1})}\right)
				+\\
				&+(-)^{\frac{k+1}{2}} \cdot k \cdot B_{k-1} \cdot 
				\iota_{X_{k-1}}	\dots \iota_{X_{0}} \omega			
				~;
			\end{split}
		\end{equation}
		where 
		\begin{equation}\label{eq:VinoMultibrakAllaZambon_2}
			\begin{split}
			\mu_k&(f_0+\alpha_0,X_1,\dots,X_{n-1}) =
			\\
			&=
			~c_k
			\sum_{1\le i<j\le k-1}(-1)^{i+j+1}\iota_{X_{k-1}}\dots   
  			\widehat{\iota_{X_{j}}}\dots \widehat{\iota_{X_{i}}}\dots
				\iota_{X_{1}} ~ [f_0+\alpha_0,X_i,X_j]_3~.
			\end{split}
		\end{equation}			
In the above formula,		$[\cdot,\cdot,\cdot]_3 = -T_0$ denotes the ternary bracket %$\mu_3$ 
associated to the untwisted ($\omega=0$) Vinogradov Algebroid, and $c_k$ is a numerical constant
		\begin{equation}\label{eq:UglyCoefficient}
			c_k= (-)^{\frac{k+1}{2}}\frac{12~B_{k-1}}{(k-1)(k-2)}.
		\end{equation}
}
%-----------------------------------------------------------%


%-----------------------------------------------------------%
\begin{frame}{Zambon's construction}
	Consider the graded manifold $T^*[r]T[1]M$.
	\vfill

	$C^\infty(T^*[r]T[1]M)$ is a \emph{$r$-Poisson algebra} (associative multiplication, degree $r$ Poisson bracket $\lbrace \cdot,\cdot\rbrace$)\footnote{Canonical Poisson bracket on the shifted cotangent bundle}
	\vfill

	$\mathcal{C}_r := C^\infty(T^*[r]T[1]M)[r]$ is a graded Lie algebra.
	\vfill

	$\mathcal{C}_r$ contains all differntial forms on $M$ since:
	\begin{displaymath}
		\begin{tikzcd}[ampersand replacement=\&,column sep = large]
			T^*[r]T[1]M \ar[r, two heads,] \& T[1]M \&[-2.5em] \\[-.5em]
			C^\infty(T^*[r]T[1]M) \& C^\infty(T[1]M) \ar[l,hook]\ar[r,phantom,"\cong"] \& \Omega(M) 
 		\end{tikzcd}
	\end{displaymath}
\end{frame}
%-----------------------------------------------------------%

%-----------------------------------------------------------%
\begin{frame}
	There exists a distinguished element $\mathcal{S}$ of degree $r+1$ in $C^\infty(T^*[r]T[1]M)$ such that $\mathcal{C}_r$ such that $\lbrace \mathcal{S},\mathcal{S}\rbrace =0 $, lifting the \emph{de Rham} differential of $M$ to $C^\infty(T^*[r]T[1]M)$ given by 
	$$ \delta := \lbrace \mathcal{S},\cdot \rbrace$$
	\vfill

	$\mathcal{S}$ has degree $1$ in $\mathcal{C}_r$ (MC element) therefore $(\mathcal{C}_r,\lbrace \cdot,\cdot\rbrace,\delta)$ is a dg Lie algebra.
	\vfill

	Consider the embedding: $$ \mathcal{C}_r^{\geq 0} \hookrightarrow \mathcal{C}_r$$
	\begin{enumerate}
		\item \cite{Getzler1991} construct a canonical $L_\infty$-algebra structure on $\mathcal{C}_r^{<0}\cong \frac{\mathcal{C}_r }{\mathcal{C}_r^{\geq 0}}$ (via \emph{derived brackets});
		\item \cite{Fiorenza2006} construct a canonical $L_\infty$-algebra associated with any dg Lie algebra morphism $\g \to \mathfrak{h}$ (via \emph{mapping cones});
		\item \cite{Pridham2010a} showed that the $L_\infty$-algebra  of 2. is a \emph{model for the homotopy fiber $\hofib(\g \to \mathfrak{h})$}.
		\item \cite{Bandiera2015} showed that 1. is a model for the homotopy fiber $\hofib(\mathcal{C}_r^{\geq 0} \hookrightarrow \mathcal{C}_r)$.
	\end{enumerate}
\end{frame}
%-----------------------------------------------------------%

%-----------------------------------------------------------%
\begin{frame}[fragile,shrink]{Details of the mapping cone construction}
	\begin{displaymath}
		\begin{tikzcd}[ampersand replacement=\&,
			/tikz/execute at end picture={
    			\node  (large) [label=right:{\it dglas},draw,rectangle, draw,dashed, fit=(A1) (A2)] {};
    			\node (large2) [label=right:{\it dgvspaces},rectangle, draw,dashed, fit=(B1) (B2)] {};}] 
			|[alias=A1]| TW(f) \ar[dr] \ar[rr] \ar[drr,phantom,near start,"\lrcorner"]\& \& 0 \ar[d] \\
			\& X \arrow[r,""{name=U, below, draw=red}] \& |[alias=A2]| Y  \\
			\\
			\& X \ar[r,""{name=D, draw=red}] \arrow[from=U, to=D, rightsquigarrow]\& |[alias=B2]| Y  \\
			|[alias=B1]| MapCoCone(f) \ar[ur] \ar[rr] \ar[rru,phantom,near start,"\urcorner"]\& \& 0 \ar[u]
		\end{tikzcd}
	\end{displaymath}
	\vfill
	\begin{enumerate}
		\item $TW(f)$ Thom-Whitney of $X\to Y$, is a model for the homotopy fiber of $X\to Y$ in dglas \cite[Def 6.1.4]{Manetti2022b};
		\item forgetting the Lie structure, $MapCoCone(f)$ is a model for the homotopy fiber of $X\to Y$ in dgvspaces \cite[Def.5.1.3]{Manetti2022b};
		\item $TW(f)$ and $MapCoCone(f)$ are quasi isomorphic as dgspaces \cite[Cor. 6.1.7]{Manetti2022b}.
		\item Via homotopy transfer one gets an $L_\infty$-structure on $MapCoCone(f)$.
		\item the corresponding $L_\infty$-algebra is quasi-isomorphic to $TW(f)$ hence is a model for the homotopy fiber of the dgla map $X\to Y$.
	\end{enumerate}
\end{frame}
%-----------------------------------------------------------%

%-----------------------------------------------------------%
\begin{frame}{What to read out of Pridham2010a}
	nell'articolo di Pridham i risultati di interesse sono il Lemma 3.24, la Proposizione 4.42 e il Corollario 4.57 (numerazione dell'ultima versione arXiv)

\end{frame}
%-----------------------------------------------------------%



%-----------------------------------------------------------%
\begin{frame}
	Consider $\mathcal{C}_r^{\geq 0}[-1] \hookrightarrow \mathcal{C}_r[-1]$
	\vfill

	Observe that $\mathcal{C}_r^{<0} = C^\infty(T^*[r]T[1]M)^{<r}=$
	\begin{displaymath}
		\begin{tikzcd}[ampersand replacement=\&]
			\Omega^0 \ar[r,"\d"]\&
			\Omega^1 \ar[r,"\d"]\&
			\Omega^2 \ar[r,"\d"]\&
			\cdots \ar[r,"\d"]\&
			\Omega^{r-2} \ar[r,"{(\d,0)}"]\&
			\Omega^{r-1}\ar[d,phantom,"\oplus"]\\[-.7em]
			\&\&\&\&\&\X
 		\end{tikzcd}
	\end{displaymath}
	\vfill

	\begin{displaymath}
		\begin{tikzcd}[ampersand replacement=\&,column sep = small]
			\& \text{\small (deg $1-r$)} \& \cdots \& \&\text{\small (deg $-1$)} \& \text{\small (deg $0$)} \\[-.5em]
			\Rogers[\sigma]{r-1}   \ar[r,phantom,":="]
			\&
			\Omega^0 \ar[r,"{\d}"] \ar[d,"id"]
			\&
			\Omega^1 \ar[r,"{\d}"] \ar[d,"id"]
			\&
			\cdots
			\&
			\Omega^{r-2} \ar[r,"{(0,\d)}"] \ar[d,"id"]
			\& \Ham_{\sigma}^{(r-1)} \ar[d,hook]
			\\
			\Courant[\sigma]{r-1} \ar[r,phantom,":="]
			\&
			\Omega^0 \ar[r,"{\d}"]
			\&
			\Omega^1 \ar[r,"{\d}"] 
			\&
			\cdots
			\&
			\Omega^{r-2} \ar[r,"{(0,\d)}"]
			\& \Gamma(E^{(1)})
		\end{tikzcd}
	\end{displaymath}
	\vfill

	\cite{Miti2021} proved that the vertical arrows are the linear part of a $L_\infty$-morphism $\Rogers[\sigma]{r-1} \to \Courant[\sigma]{r-1}$.
\end{frame}
%-----------------------------------------------------------%




%-----------------------------------------------------------%
\ifstandalone
% https://en.wikibooks.org/wiki/LaTeX/Bibliographies_with_biblatex_and_biber
\begin{frame}[t,allowframebreaks]{Partial Bibliography}
	\nocite{Miti2021}
	\bibliographystyle{alpha}
	\bibliography{bibfile}
\end{frame}
\fi
%-----------------------------------------------------------%

\end{document}
%-----------------------------------------------------------%

%-----------------------------------------------------------%
\section{The sought Lie infinity morphism}
%-----------------------------------------------------------%
	\checkpoint	
	%- HandOut Flag -----------------------------------------------------------------------------------------
\makeatletter
\@ifundefined{ifHandout}{%
  \expandafter\newif\csname ifHandout\endcsname
}{}
\makeatother

%- D0cum3nt ----------------------------------------------------------------------------------------------
\documentclass[beamer,10pt]{standalone}   
%\documentclass[beamer,10pt,handout]{standalone}  \Handouttrue  

\ifHandout
	\setbeameroption{show notes} %print notes   
\fi

	
%- Packages ----------------------------------------------------------------------------------------------
\usepackage{custom-style}
\usepackage{math}




%--Beamer Style-----------------------------------------------------------------------------------------------
\usetheme{toninus}
\usepackage{animate}
\usetikzlibrary{positioning, arrows}
\usetikzlibrary{shapes}

%===========================================================%
\begin{document}
%===========================================================%
%\checkpoint
%-----------------------------------------------------------%
\subsection{From twisted to untwisted higher Courant algebras}
%-----------------------------------------------------------%

%-----------------------------------------------------------%
\begin{frame}{From twisted to untwisted higher Courant algebras (1)}
	%
	\begin{block}{Goal:}
		Exhibit an $L_\infty$-morphism
		$$ \Courant[\sigma]{r-1} \to \Courant{r}$$
	\end{block}
	\vfill \pause	

	\begin{exblock}[Lowest case ]
	For $r=1,2$ we have:
	\begin{displaymath}
		\begin{tikzcd}[ampersand replacement=\&,column sep = small]
			\Courant[\sigma]{0} \ar[r,phantom,":="] \ar[d]
			\&[1em]
			0 \ar[r] \ar[d]
			\& \Omega^0\oplus \X \ar[d,"{(\d,\id)}"]
			\\
			\Courant{1} \ar[r,phantom,":="] 
			\&
			\Omega^0 \ar[r,"{(\d,0)}"] 
			\& 
			\Omega^1\oplus \X
		\end{tikzcd}
	\end{displaymath}
	\cite{Zambon2012} proved that  the vertical arrows are the linear part of a $L_\infty$-morphism $\Courant[\sigma]{0} \to \Courant[\sigma]{1}$.
	\end{exblock}
	\vfill
	
\end{frame}
\note[itemize]
{
	\item 
}

%-----------------------------------------------------------%
\begin{frame}{From twisted to untwisted higher Courant algebras (2)}
	%
	More generally we have a morphism of chain complexes:
	\begin{displaymath}
		\begin{tikzcd}[ampersand replacement=\&,column sep = small]
			\Courant[\sigma]{r-1} %\ar[r,phantom,":="] \ar[d]
			\&[1em]
			0 \ar[r] \ar[d]
			\& \Omega^0 \ar[d,"\d"] \ar[r]
			\& \Omega^1 \ar[d,"\d"] \ar[r]
			\& \cdots \ar[r]
			\& \Omega^{r-2} \ar[d,"\d"] \ar[r,"{(\d,0)}"]
			\&[.5em]  \Omega^{r-1} \oplus \X \ar[d,"{(\d,\id)}"]
			\\
			\Courant{r} %\ar[r,phantom,":="]
			\& \Omega^0 \ar[r]
			\& \Omega^1 \ar[r]
			\& \Omega^2 \ar[r]
			\& \cdots \ar[r]
			\& \Omega^{r-1} \ar[r,"{(\d,0)}"]
			\&\Omega^{r} \oplus \X
		\end{tikzcd}
	\end{displaymath}

	\vfill\pause

	\begin{thmblock}[{[Fiorenza-M. 2025]}]
		The above chain complex morphism is the linear part of a $L_\infty$-morphism $$\Courant[\sigma]{r} \to \Courant{r+1}~.$$
	\end{thmblock}

	\vfill
\end{frame}
\note[itemize]
{
	\item .
}
%-----------------------------------------------------------%

%-----------------------------------------------------------%
\begin{frame}{Idea of the proof:}
	%
	Recall that $$\Courant[\sigma]{r-1} = \hofib(\mathcal{C}_{r,\sigma}^{\geq 0} \hookrightarrow \mathcal{C}_{r,\sigma}) $$
	\pause
	\vfill

	We want to exhibit a morphism:
		\begin{displaymath}
			\begin{tikzcd}[ampersand replacement=\&]
				\hofib(\mathcal{C}_{r,\sigma}^{\geq 0} \hookrightarrow \mathcal{C}_{r,\sigma}) \ar[d]
				\\
				\hofib(\mathcal{C}_{r+1}^{\geq 0} \hookrightarrow \mathcal{C}_{r+1})
			\end{tikzcd}
		\end{displaymath}
	\pause
	\vfill

	This should be naturally\footnote{By the universal property of homotopy limits.} induced by a (homotopy) commutative diagram of dg-Lie algebras (or $L\infty$-algebras):
		\begin{displaymath}
			\begin{tikzcd}[ampersand replacement=\&]
				\mathcal{C}_{r,\sigma}^{\geq 0} \ar[r,hook] \ar[d ]
				\& \mathcal{C}_{r,\sigma} \ar[d, ]
				\\
				\mathcal{C}_{r+1}^{\geq 0} \ar[r,hook] \& \mathcal{C}_{r+1} 
			\end{tikzcd}
		\end{displaymath}
	
\end{frame}
\note[itemize]
{
	\item 
	\item 
}
%-----------------------------------------------------------%

%-----------------------------------------------------------%
\begin{frame}{A technical lemma (1)}
  Inside $\mathcal{C}_r$ there is a smaller, mora manageble, dg-Lie subalgebra with the same negative part.
  \vfill \pause

  \begin{defblock}[Cartan dg-Lie algebra]
  	Is a $2$-terms dg-Lie algebra $\Cartan \subseteq \Der(\Omega)$ given by:
  \begin{displaymath}
	\begin{tikzcd}[ampersand replacement=\&,row sep = small]
		\text{\tiny (deg $-1$)} \& \text{\tiny (deg $0$)}\&[2em]
		\\[-.7em]
		\iota_{\X} \ar[r,"\delta"] \& \Lie_{\X} \ar[r,phantom,";"] \& \delta:= [\d, \cdot]
	\end{tikzcd}
  \end{displaymath}
  \end{defblock}
  \vfill \pause

  $\Omega$ is a module for $\Der(\Omega)$ and so it is a module for $\Cartan$.
  \\
  $\Rightarrow$\quad By shifting degrees, $\Omega[r]$ is a module for $\Cartan$.
  \vfill 

\end{frame}
\note[itemize]
{
	\item 
	\item 
}
%-----------------------------------------------------------%

%-----------------------------------------------------------%
\begin{frame}{A technical lemma (2)}
	  \begin{defblock}[The \emph{small} dgla]
		We can form the dg-Lie algebra
		$$ \g_r := \Cartan \ltimes \Omega[r]$$
		\pause
		\smallskip

		Observe that any $\sigma \in \Omega^{r+1}_{\mathrm{cl}})$ is a MC (degree 1) element of $\g_r$.
		\pause
		\smallskip
		
		We can consider the twisted dg-Lie algebra $(\g_{r,\sigma}, \d_\sigma, \lbrace \cdot,\cdot\rbrace_\sigma)$ with:
	\begin{displaymath}
		\d_\sigma (\omega) = \d \omega~, \qquad
		\d_\sigma (\iota_X) = \Lie_X + \iota_X \sigma~, \qquad
		\d_\sigma(\Lie_X) = -\Lie_X\sigma
	\end{displaymath}
	\end{defblock}
	\vfill\onslide<4->{

	\begin{lemblock}[{$\hofib(\mathcal{C}_{r,\sigma}^{\geq 0} \hookrightarrow \mathcal{C}_{r,\sigma}) \cong \hofib(\g_{r,\sigma}^{~\geq 0} \hookrightarrow \g_{r,\sigma})$}]

			We have\footnote{the left square is in DGLAs, the right one is in dg vectors spaces}:
			\vspace{-1em}
	\begin{displaymath}
		\begin{tikzcd}[ampersand replacement=\&]
			\g_{r,\sigma}^{~\geq 0} \ar[r,hook] \ar[d,hook] \&
			\g_{r,\sigma} \ar[d,hook] \ar[r,two heads] \&[1em]
			\g_{r,\sigma}^{~< 0} \ar[d,equal] \\
			\mathcal{C}_{r,\sigma}^{~\geq 0} \ar[r,hook] \&
			\mathcal{C}_{r,\sigma} \ar[r,two heads] \&
			\mathcal{C}_{r,\sigma}^{~< 0}
		\end{tikzcd}
	\end{displaymath}

	Therefore $\mathcal{C}_{r,\sigma}^{~< 0}[-1] = \g_{r,\sigma}^{~<0}[-1]$ as $L_\infty$-algebras with the \cite{Getzler1991} derived brackets.
	\end{lemblock}
	}
	\vfill

\end{frame}
\note[itemize]
{
	\item .
}
%-----------------------------------------------------------%

%-----------------------------------------------------------%
\begin{frame}{A technical lemma (3)}
	\begin{upshotblock}
	  Replace $\mathcal{C}_{r,\sigma}, \mathcal{C}_{r+1}$ with $\g_{r,\sigma}, \g_{r+1}$ in the previous commuative square:
		\begin{displaymath}
			\begin{tikzcd}[ampersand replacement=\&]
				\g_{r,\sigma}^{~\geq 0} \ar[r,hook] \ar[d, ]
				\& \g_{r,\sigma} \ar[d, ]
				\\
				\g_{r+1}^{~\geq 0} \ar[r,hook] \& \g_{r+1}
			\end{tikzcd}
		\end{displaymath}
		\medskip

		to naturally obtain a morphism $\Courant[\sigma]{r-1} \to \Courant{r}$.
	\end{upshotblock}
	\vfill

	Moreover, if this has to be true for any $\sigma \in \Omega^{r+1}_{\mathrm{cl}}$, it has to be true for $\sigma = 0$.... 
\end{frame}
\note[itemize]
{
	\item .
}
%-----------------------------------------------------------%

%-----------------------------------------------------------%
\begin{frame}{THM proof: untwisted case (1)}
	Consider the simpler situation $\sigma=0$:
	\begin{displaymath}
		\begin{tikzcd}[ampersand replacement=\&]
			\g_{r}^{~\geq 0} \ar[r,hook] \ar[d, ]
			\& \g_{r} \ar[d,"\Phi_1"]
			\\
			\g_{r+1}^{~\geq 0} \ar[r,hook] \& \g_{r+1}
		\end{tikzcd}
	\end{displaymath}
	\vfill \pause

	\begin{itemize}[<+-|@alert@+>]
		\item $\Phi_1:\g_{r,\sigma}\to \g_{r+1}$ is a morphism of cochain complexes:
		\begin{displaymath}
		\begin{tikzcd}[ampersand replacement=\&, column sep = small]
			0\ar[r]\ar[d] \&
			\Omega^0 \ar[r] \ar[d,"\d"] \&
			\Omega^1 \ar[r] \ar[d,"\d"] \&
			\Omega^2 \ar[r] \&
			\cdots \ar[r] \&
			\Omega^{r-1}\oplus \iota_{\X} \ar[d,"\d\oplus \id"] \ar[r] \&
			\Omega^{r}\oplus \Lie_{\X} \ar[r]\ar[d,"\d\oplus\id"] \&
			\Omega^{r+1} \ar[d,"\d"] \ar[r,phantom,"\cdots"] \& \phantom{.}
			\\
			\Omega^{0} \ar[r] \&
			\Omega^{1} \ar[r] \&
			\Omega^{2} \ar[r] \&
			\cdots \& \cdots \ar[r] \&
			\Omega^{r}\oplus\iota_{\X} \ar[r] \&
			\Omega^{r+1}\oplus \Lie_{\X} \ar[r] \&
			\Omega^{r+2} \ar[r,phantom,"\cdots"]  \& \phantom{.}
		\end{tikzcd}
		\end{displaymath}
		\vfill

		\item Is $\Phi_1$ a dg-Lie algebra morphism?
			$$ \Phi_1\big(\lbrace a, b \rbrace\big) \overset{?}{=} \big\lbrace \Phi_1(a), \Phi_1(b) \big\rbrace$$
		\vfill

		\item 	No: the de Rham differential does not commute with the contractions:
		$$ \Phi_1\big(\lbrace \iota_X, \omega \rbrace\big) \neq \big\lbrace \Phi_1(\iota_X), \Phi_1(\omega)\big\rbrace~.$$
	\end{itemize}
\end{frame}
\note[itemize]
{
	\item  
	\item 
}
%-----------------------------------------------------------%

%-----------------------------------------------------------%
\begin{frame}{THM proof: untwisted case (2)}
	We can cure this discrepancy considering an higher term:
	\begin{align*}
		\Phi_2(\iota_X,\omega) &=~ \iota_X \omega \\
		\Phi_2(\omega,\iota_X) &=~ \pm \iota_X \omega \\
		\Phi_2(a,b) &=~ 0 \qquad \text{in all other cases}
	\end{align*}
	\vfill\pause

	\begin{itemize}
		\item  $\Phi= (\Phi_1,\Phi_2,0,\dots)$ is a $L_\infty$-morphism $\g_r \to \g_{r+1}$.
		\pause\vfill
		\item $\Phi_1$  maps $\g_{r,\sigma}^{~\geq 0}$ to $\g_{r+1}^{~\geq 0}$.
		\only<3->{\blfootnote{ We need to check that $\Phi_2(a,b)\in \g_{r+1}^{~\geq 0}$ when $|a|=|b|=0$. True since  $\Phi_2(a,b)=0$ for these elements.}}
	\end{itemize}
	\vfill\pause

	\begin{upshotblock}
		We have a strictly commutative diagram of $L_\infty$-morphisms:
		\begin{displaymath}
			\begin{tikzcd}[ampersand replacement=\&]
				\g_{r,\sigma}^{~\geq 0} \ar[r,hook] \ar[d, "\Phi \vert_{\geq 0}"']
				\& \g_{r} \ar[d, "\Phi"]
				\\
				\g_{r+1}^{~\geq 0} \ar[r,hook] \& \g_{r+1}
			\end{tikzcd}
		\end{displaymath}
	\end{upshotblock}

\end{frame}
\note[itemize]
{
	\item .
}
%-----------------------------------------------------------%

%-----------------------------------------------------------%
\begin{frame}{THM proof: twisted case}
  What for arbitrary $\sigma \in \Omega^{r+1}_{\mathrm{cl}}$? $\quad\Rightarrow \quad$
  Use $\sigma$ to twist everything:
  \vfill \pause	

	\begin{displaymath}
		\begin{tikzcd}[ampersand replacement=\&]
			\g_{r,\sigma}^{~\geq 0} \ar[r,hook] \ar[d, "\Phi_\sigma \vert_{\geq 0}"']
			\& \g_{r,\sigma} \ar[d, "\Phi_\sigma"]
			\\
			\g_{r+1\only<1-3>{,\Phi(\sigma)}}^{~\geq 0} \ar[r,hook] \& \g_{r+1,\only<1-3>{\Phi(\sigma)}}
		\end{tikzcd}
	\end{displaymath}
	\vfill\pause

	Where:
	\begin{align*}\small
		\big(\Phi_\sigma \big)_j (\cdots) 
		&=~
		\sum_{k=0}^\infty \frac{\Phi_{k+j}}{k!} (\overbrace{\sigma,\cdots,\sigma}^{\text{\tiny $k$ copies}},\cdots)
		\\
		\onslide<4->{
		\Phi(\sigma) &=~ \sum_{k=0}^\infty \frac{\Phi_k(\sigma,\cdots,\sigma)}{k!} = \d(\sigma) + \frac{1}{2}\cdot 0 = 0}
	\end{align*}
	\vfill

	\only<4->{
	\blfootnote{
		We have to check that $\Phi_\sigma$ preseves the positive part. This is immediate since $(\Phi_\sigma)_2 = \Phi_2$.\\ 
		Moreover, we have 
			$$ (\Phi_\sigma)_1 \big\vert_{\geq 0} = \Phi_1 \big\vert_{\geq 0} + \frac{1}{2}\cancel{\Phi_2(\sigma,\cdot)\big\vert_{\geq 0}} = \Phi_1\big\vert_{\geq 0} $$
	}
	}

\end{frame}
\note[itemize]
{
	\item .
}
%-----------------------------------------------------------%


%-----------------------------------------------------------%
\subsection{From Rogers to Courant algebras}
%-----------------------------------------------------------%

%-----------------------------------------------------------%
\begin{frame}{The \cite{Miti2024} $L_\infty$-morphism revisited (1)}
	Recall:
	\begin{displaymath}
		\g_{r,\sigma}^{<0} = \left(
		\begin{tikzcd}[ampersand replacement=\&]
			\Omega^0 \ar[r,"\d"]\&
			\Omega^1 \ar[r,"\d"]\&
			\Omega^2 \ar[r,"\d"]\&
			\cdots \ar[r,"\d"]\&
			\Omega^{r-2} \ar[r,"{(\d,0)}"]\&
			\Omega^{r-1}\ar[d,phantom,"\oplus"]\\[-.7em]
			\&\&\&\&\&\X
 		\end{tikzcd}
		\right)
	\end{displaymath}
	\vfill\pause

	\begin{displaymath}
		\begin{tikzcd}[ampersand replacement=\&,column sep = small]
			\& \text{\tiny (deg $1-r$)} \& \cdots \& \&\text{\tiny (deg $-1$)} \& \text{\tiny (deg $0$)} \\[-1.5em]
			\Rogers[\sigma]{r-1}   \ar[r,phantom,":="]
			\&
			\Omega^0 \ar[r,"{\d}"] \ar[d,"id"]
			\&
			\Omega^1 \ar[r,"{\d}"] \ar[d,"id"]
			\&
			\cdots
			\&
			\Omega^{r-2} \ar[r,"{(0,\d)}"] \ar[d,"id"]
			\& \Ham_{\sigma}^{(r-1)} \ar[d,hook]
			\\
			\Courant[\sigma]{r-1} \ar[r,phantom,":="]
			\&
			\Omega^0 \ar[r,"{\d}"]
			\&
			\Omega^1 \ar[r,"{\d}"] 
			\&
			\cdots
			\&
			\Omega^{r-2} \ar[r,"{(0,\d)}"]
			\& \Gamma(E^{(1)})
		\end{tikzcd}
	\end{displaymath}
	\vfill\pause

	\begin{thmblock}[{[M.-Zambon 2024]}]
		The vertical arrows are the linear part of a $L_\infty$-morphism $$\Rogers[\sigma]{r-1} \to \Courant[\sigma]{r-1}.$$
	\end{thmblock}
	\vfill
\end{frame}
\note[itemize]
{
	\item .
}
%-----------------------------------------------------------%



%-----------------------------------------------------------%
\begin{frame}{The \cite{Miti2024} $L_\infty$-morphism revisited (2)}
	\begin{thmblock}[{[Fiorenza-M. 2025]}]
	\cite{Miti2024} morphism comes from an $L_\infty$ commutative square:
	\begin{displaymath}
		\begin{tikzcd}[ampersand replacement=\&]
			\Rogers[\sigma]{r-1} \ar[d] \ar[r] \&
			\X_{\ham,\sigma} \ar[r,"\iota_{\dots}\sigma"] \ar[d,"\Lie"]\&
			(\Omega^0\to\Omega^1\to\cdots\to\Omega^{r-1}\to \d \Omega^{r-1}) \ar[d,hook] 
			\\
			\Courant[\sigma]{r-1} \ar[r] \&
			\g_{r+1,\sigma}^{~\geq 0} \ar[r,hook] \ar[ur,Rightarrow]\& 
			\g_{r+1,\sigma}
		\end{tikzcd}
	\end{displaymath}
	\end{thmblock}
	\vfill\pause

	Since \cite{Fiorenza2014a} shows that:
	\begin{itemize}
		\item[$\bullet$] $\iota_{\ldots}\sigma$ (multicontractions with $\sigma$) is a $L_\infty$-morphism
		\item[$\bullet$] $\Rogers[\sigma]{r-1} =\hofib(\iota_{\ldots}\sigma)$ \quad {\color{gray} \small (is a model of...)\color{black}}
	\end{itemize}
	\vfill\pause

	The homotopy is given by $$ e^{\iota} * \Lie = \iota_{\ldots}\sigma$$
	which is basically Cartan's magic formula for multivector fields.
	$$[\d, \iota_{x_1\wedge\cdots\wedge x_k}] = \Lie_{x_1\wedge\cdots\wedge x_k}$$
\end{frame}
\note[itemize]
{
	\item The multicontractions $\iota_{x_1\wedge\cdots\wedge x_k}\sigma$ appear tipical of Roger's algebra appear by twisting the differential, i.e. changing $\d$ in $\d_\sigma$.
}

%-----------------------------------------------------------%

 


%-----------------------------------------------------------%
\ifstandalone
% https://en.wikibooks.org/wiki/LaTeX/Bibliographies_with_biblatex_and_biber
\begin{frame}[t,allowframebreaks]{Partial Bibliography}
	\nocite{Miti2021}
	\bibliographystyle{alpha}
	\bibliography{bibfile}
\end{frame}
\fi
%-----------------------------------------------------------%

\end{document}
%-----------------------------------------------------------%



%-------------------------------------------------------------------------------------------------------------------------------------------------
	\addtocounter{framenumber}{-1}
		\begin{frame}{Conclusion:}
			\vfill
		  \centering 
		  \onslide<2->{\Huge\color{red} 
		  \emph{Thank you for your attention!}\normalsize \color{black}}
			\vfill
			%
			\centering
				\begin{displaymath}
					\begin{tikzcd}[ampersand replacement=\&]
						\Rogers[\sigma]{r-1} \ar[d] \ar[r] \&
						\X_{\ham,\sigma} \ar[r,"\iota_{\dots}\sigma"] \ar[d,"\Lie"]\&
						(\Omega^0\to\Omega^1\to\cdots\to\Omega^{r-1}\to \d \Omega^{r-1}) \ar[d,hook] 
						\\
						\Courant[\sigma]{r-1} \ar[r] \ar[d]\&
						\g_{r+1,\sigma}^{~\geq 0} \ar[r,hook] \ar[ur,Rightarrow] \ar[d,"\Phi\big\vert_{\geq 0}"]\& 
						\g_{r+1,\sigma} \ar[d,"\Phi"]
						\\
						\Courant{r} \ar[r] \&
						\g_{r}^{\geq 0} \ar[r,hook]\&
						\g_{r}
				\end{tikzcd}
			\end{displaymath}
		\end{frame}
		\note[itemize]{
			\item
		}
%------------------------------------------------------------------------------------------------



%------------------------------------------------------------------------------------------------
% APPENDIX
%------------------------------------------------------------------------------------------------
\appendix
%-------------------------------------------------------------------------------------------------------------------------------------------------
\section{Complementary Material}
	%+----------------------------------------------------------------------------+
%| SLIDES: 
%| Chapter: Complementary material - details on eventual questions
%| Author: Antonio miti
%| Event: PHD preliminary Defence
%+----------------------------------------------------------------------------+

%- HandOut Flag -----------------------------------------------------------------------------------------
\newif\ifHandout

%- D0cum3nt ----------------------------------------------------------------------------------------------
\documentclass[beamer,10pt]{standalone}   
%\documentclass[beamer,10pt,handout]{standalone}  \Handouttrue  

%- HandOut Flag -----------------------------------------------------------------------------------------
\ifHandout
	\setbeameroption{show notes} %print notes   
\fi

	
%- Packages ----------------------------------------------------------------------------------------------
\usepackage{custom-style}
\usepackage{math}

%--Beamer Style-----------------------------------------------------------------------------------------------
\usetheme{toninus}



\providecommand{\blank}{\text{\textvisiblespace}}


\newcommand{\subsectiontitle}{
  \begin{frame}
  \vfill
  \centering
  \begin{beamercolorbox}[sep=8pt,center,shadow=true,rounded=true]{title}
    \usebeamerfont{title}\insertsectionhead\par%
    \usebeamerfont{title}\insertsubsectionhead\par%
  \end{beamercolorbox}
  \vfill
  \end{frame}
}

\providecommand{\blank}{\text{\textvisiblespace}}




%===========================================================%

%- D0cum3nt %===========================================================%

\begin{document}
%===========================================================%


%##################################################################################
\begin{frame}
	\begin{center}
	\Huge\emph{Supplementary Material}
	\end{center}
\end{frame}
\note[itemize]{
	\item
}
\addtocounter{framenumber}{-1}
%##################################################################################





%===================================================================================
\section{Background}
%===================================================================================



%-----------------------------------------------------------%
\subsection{Symplectic Manifolds}
%-----------------------------------------------------------%
 


%-----------------------------------------------------------%
\begin{frame}[t, fragile]{Research Framework:  \textbf{multisymplectic geometry}} %Fragile -->workaround tikzcd
	\begin{center}
		$-$ \emph{multisymplectic means \textbf{going higher} in the degree of $\omega$} $-$
	\end{center}
	\pause
	\begin{defblock}[$n$-plectic manifold ~\emph{(Cantrijn, Ibort, De Le\'on)} \cite{Cantrun2017}]
		\includestandalone[width=0.95\textwidth]{./Pictures/Figure_multisym}	
	\end{defblock}
	%
	\vfill
	%
	%
	\pause
	\begin{block}{Examples:}
		\begin{itemize}
			\item[$\bullet$] 1-plectic $=$ symplectic
			\item[$\bullet$] Any oriented $(n+1)$-dimensional manifold is $n$-plectic w.r.t. the volume form.
			\item[$\bullet$] The multicotangent bundle $\Lambda^n T^\ast Q$ is naturally $n$-plectic.
		\end{itemize}
	\end{block}			 
%
	\pause
	\begin{block}{Historical motivation}
		Mechanics: geometrical foundations of \textit{(first-order)} field theories.
		\begin{itemize}
		 \item[•] Kijowski, W. Tulczyjew \cite{Kijowski1979}; %(1979)
		 \item[•] Cariñena, Crampin, Ibort \cite{Carinena1991b};% (1991)
		 \item[•] Gotay, Isenberg, Marsden, Montgomery \cite{Gimmsy1};%(1998)
		 \\ $\cdots$
		\end{itemize}
	\end{block}
\end{frame}
%-----------------------------------------------------------%


%-----------------------------------------------------------%
\begin{frame}{Observables in \textbf{multisymplectic geometry}}
	%
	\begin{defblock}[Hamiltonian $(n-1)$-forms]
		\begin{displaymath}
			\Omega^{n-1}_{ham}(M,\omega) 	:=
			\biggr\{ \sigma \in  \Omega^{n-1}(M) \; \biggr\vert \; 
				\exists \vHam_\sigma \in \mathfrak{X}(M) ~:~ 
				\tikz[baseline,remember picture]{\node[rounded corners,
                        fill=orange!5,draw=orange!30,anchor=base]            
            			(target) {$d \sigma = -\iota_{\vHam_\sigma} \omega$ };
            	}				
				~\biggr\} 
			\end{displaymath}
	\end{defblock}
	%
	\onslide<2>{
		\tikz[overlay,remember picture]
		{
			\node[rounded corners,
                 fill=orange!5,draw=orange!30,anchor=base]
            	 (base) at ($(current page.north east)-(2,1)$) [rotate=-0,text width=3.5cm,align=center] {\footnotesize{\textcolor{red}{Hamilton-DeDonder-Weyl \\equation}}};
		}	
		\begin{tikzpicture}[overlay,remember picture]
		    	\path[->] (base.south east) edge[bend left,red](target.east);
	    \end{tikzpicture}
	}
	%
	\vspace{-1em}
	\pause
	\begin{columns}[T]
		\setlength{\belowdisplayskip}{5pt}
		\begin{column}{.50\linewidth}
			%
			\centering \it
			$-$ symplectic case $-$
			\onslide<3->{
			\begin{thmblock}[Observables Poisson algebra]
				$C^\infty(M,\omega)$ endowed with
				\vspace{-.5em}
				\begin{displaymath}
					\lbrace \sigma_1, \sigma_2 \rbrace =			
					~ - \iota_{\vHam_1}\iota_{\vHam_2} \omega 
					~= \mathcal{L}_{\vHam_1} \sigma_2
				\end{displaymath}			
				forms a Poisson algebra.
			\end{thmblock}
			}
			%
			\onslide<4->{
			\vspace{1em}
			\begin{itemize}
				\item[\cmark] Skew-symmetric;
				\item[\cmark] multiplication of observables;
				\item[\cmark] Leibniz Rule;
				\item[\cmark] Jacobi equation;
			\end{itemize}		
			}		
		\end{column}	
		%
		\onslide<1->{\vrule{}}
		%
		\begin{column}{.50\linewidth}
			\centering \it
			$-$ $n$-plectic case $-$
			\onslide<5->{			
			\begin{thmblock}[Observables $L_\infty$-algebra]
				$\Omega^{n-1}_{ham}(M,\omega)$ endowed with
				\vspace{-.5em}
				\begin{displaymath}
					\lbrace \sigma_1, \sigma_2 \rbrace =			
					~ - \iota_{\vHam_1}\iota_{\vHam_2} \omega 
				\end{displaymath}			
				can be extended to a \\ $L_\infty-algebra$.
			\end{thmblock}
			}
			%
			\onslide<6->{
			\begin{itemize}
				\item[\cmark] Skew-symmetric;
				\item[\xmark] multiplication of observables;
				\item[\xmark] Jacobi equation;
				%\\ \hspace*{4.25em} full-fledged Jacobi equation;
				\item[\smark] Jacobi equation \emph{up to homotopies}.
			\end{itemize}			
			}
		\end{column}	
	\end{columns}
\end{frame}
%-----------------------------------------------------------%

%-----------------------------------------------------------%
\begin{frame}[fragile]{Lie $\infty$-algebra of Observables (higher observables) }
	Let be $(M,\omega)$ a $n$-plectic manifold.
	  	\vfill
	\begin{defblock}[$L_\infty$-algebra of observables ~\emph{(Rogers)} \cite{Rogers2010}]
		\medskip
		\hspace{.25em} Is a cochain-complex $(L,\{\cdot\}_1)$ \\
		\vspace{-1em}
		\begin{center}
			\includestandalone[width=0.95\textwidth]{Pictures/Frame_Observables}
		\end{center}
		\onslide<2->{
			\bigskip
			\hspace{.25em} with $n$ (skew-symmetric) multibrackets $(2 \leq k \leq n+1)$\\
			\vspace{-1em}
			\onslide<3->{
				\begin{center}
					\includestandalone{Pictures/Equation_Multibracket}	
				\end{center}
			}
			\medskip
		}
		%
	\end{defblock}
  \end{frame}
%-----------------------------------------------------------%

%-----------------------------------------------------------%
\begin{frame}[fragile,shrink]{Details of the mapping cone construction \cite{Fiorenza2006}}
	\begin{displaymath}
		\begin{tikzcd}[ampersand replacement=\&,
			/tikz/execute at end picture={
    			\node  (large) [label=right:{\it dglas},draw,rectangle, draw,dashed, fit=(A1) (A2)] {};
    			\node (large2) [label=right:{\it dgvspaces},rectangle, draw,dashed, fit=(B1) (B2)] {};}] 
			|[alias=A1]| TW(f) \ar[dr] \ar[rr] \ar[drr,phantom,near start,"\lrcorner"]\& \& 0 \ar[d] \\
			\& X \arrow[r,""{name=U, below, draw=red}] \& |[alias=A2]| Y  \\
			\\
			\& X \ar[r,""{name=D, draw=red}] \arrow[from=U, to=D, rightsquigarrow]\& |[alias=B2]| Y  \\
			|[alias=B1]| MapCoCone(f) \ar[ur] \ar[rr] \ar[rru,phantom,near start,"\urcorner"]\& \& 0 \ar[u]
		\end{tikzcd}
	\end{displaymath}
	\vfill
	\begin{enumerate}
		\item $TW(f)$ Thom-Whitney of $X\to Y$, is a model for the homotopy fiber of $X\to Y$ in dglas \cite[Def 6.1.4]{Manetti2022b};
		\item forgetting the Lie structure, $MapCoCone(f)$ is a model for the homotopy fiber of $X\to Y$ in dgvspaces \cite[Def.5.1.3]{Manetti2022b};
		\item $TW(f)$ and $MapCoCone(f)$ are quasi isomorphic as dgspaces \cite[Cor. 6.1.7]{Manetti2022b}.
		\item Via homotopy transfer one gets an $L_\infty$-structure on $MapCoCone(f)$.
		\item the corresponding $L_\infty$-algebra is quasi-isomorphic to $TW(f)$ hence is a model for the homotopy fiber of the dgla map $X\to Y$.
	\end{enumerate}
\end{frame}
%-----------------------------------------------------------%

%-----------------------------------------------------------%
\begin{frame}[t]{The algebra of functions of $T[1]M$ is de Rham complex of $M$}
	\begin{claimblock}
		$$ C^\infty(T[1]M) \cong \Omega(M) ~.$$
	\end{claimblock}
	\begin{itemize}
		\item[$\bullet$] By $T[1]M$ we mean the split \cite{Bonavolont2013} $\mathbb{N}$-graded manifold associated with the \emph{graded vector bundle} $TM[1]\to M$ whose fibers, concentrated in $deg -1$ are given by the tangent spaces shifted by $1$;
		\vfill
		\item[$\bullet$] The algebra of functions of a split graded manifold corresponding to the $\mathbb{N}$-graded vector bundle $E$ are by definition given by $$\Gamma(S(E^*)).$$
		\vfill
		\item[$\bullet$] Therefore the algebra of functions of $T[1]M$ is the space of sections of the symmetric algebra of the dual bundle $S(TM[1]^*)$. I.e.
		\begin{displaymath}
			C^\infty(T[1]M) = \Gamma(S(TM[1]^*)) = \Gamma(\Lambda(TM^*)) = \Omega(M)~.
		\end{displaymath}
	\end{itemize}
	\begin{flushright}
		\qedsymbol
	\end{flushright}
\end{frame}
%-----------------------------------------------------------%

\begin{frame}{Defining Graded Manifolds. (credits to \href{https://www.ryvkin.eu/}{L.Ryvkin})}
	The following controvariant functos are fully faithful:
	\begin{columns}
		\begin{column}{.5\linewidth}
			\begin{displaymath}
				\lambdamorphism{\cat{Mfd}}{\cat{CAlg}}{M}{C^\infty(M)}
			\end{displaymath}
		\end{column}
		\begin{column}{.5\linewidth}
			\begin{displaymath}
				\lambdamorphism{\cat{VecBun}}{C^\infty-\cat{Mod}}{E}{\Gamma(E^\star)}
			\end{displaymath}
		\end{column}
	\end{columns}
	\vfill

	Consider
	\begin{itemize}
		\item[$\bullet$] $\cat{gCAlg}$ the caty. of $\mathbb{Z}$-graded commuative (asso. unital) algebras;
		\item[$\bullet$] $\cat{dgCAlg}$ the caty. of differential $\mathbb{Z}$-graded commuative algebras:
		\begin{displaymath}
			\cat{dgCAlg} := \left \lbrace
				(A,Q) ~\vert~
				A \in \cat{gCAlg}, Q \in \Der_1(A), Q^2 = 0~.
			\right\rbrace
		\end{displaymath} 
	\end{itemize}
	\vfill

	Let:
	\begin{itemize}
		\item[$\bullet$] $\cat{gVecBun}$ the caty. of $\mathbb{Z}$-graded vectord bundle (degree-wise finite dimensionaL);
	\end{itemize}
	\vfill

	\begin{defblock}[Caty. of graded manifolds]
		$\cat{gMfd}$ is the essential image\footnote{i.e. $\mathcal{M}\in \cat{gCAlg}^{op}$ s.t. $\exists E \in \cat{gVecBun}$ with $\mathcal{M}=\Gamma(S(E^*))$.} of the functor
		\begin{displaymath}
			\lambdamorphism{\cat{gVecBun}}{\cat{gCAlg}^{op}}{E}{\Gamma(S(E^*))}
		\end{displaymath}
	\end{defblock}
\end{frame}

\begin{frame}{Examples of Graded Manifolds (credits to \href{https://www.ryvkin.eu/}{L.Ryvkin})}
	For $ \mathcal{M}\in \cat{gCAlg}^{\mathrm{op}}$, we call $C^\infty(\mathcal{M})$ the corresponding object in $\cat{gCAlg}$

	\begin{exblock}[ of graded manifolds]
		\begin{table}
		\begin{tabular}{|l|l|l|}
			\hline
			\textbf{Graded Manifold} $\mathcal{M}$ & \textbf{Algebra of functions} $C^\infty(\mathcal{M})$
			\\
			\hline
			$T[1]M$ & $\Omega(M)$ & $M$ manifold\\
			$T^\ast[1]M$ & $\X^{\bullet}(M)$ & " \\
			$A[1]$ & $\Gamma(\Lambda A^*) = \Gamma(S(A_{-1})^*)$ & $A\in \cat{VecBun}$ \\
			$\g[1]$ & $\Lambda \g^*$ & $\g$ vec. space (e.g. Lie algebra) \\
			$T[1]\g[1]$ & $S \g^* \otimes \Lambda \g^*$ & " \\
			\hline
		\end{tabular}
		\end{table}		
	\end{exblock}


\end{frame}


%-----------------------------------------------------------%
\ifstandalone
% https://en.wikibooks.org/wiki/LaTeX/Bibliographies_with_biblatex_and_biber
\begin{frame}[t,allowframebreaks]{Partial Bibliography}
	\nocite{Miti2021}
	\bibliographystyle{alpha}
	\bibliography{bibfile}
\end{frame}
\fi
%-----------------------------------------------------------%


%===========================================================%
\end{document}
%===========================================================%

	




%- HandOut Flag -----------------------------------------------------------------------------------------
\newif\ifHandout
%	\Handouttrue  %uncomment for the printable version

%- D0cum3nt ----------------------------------------------------------------------------------------------
\documentclass[beamer,handout,10pt]{standalone}   
\ifHandout
	\setbeameroption{show notes} %print notes   
\fi

	
%- Packages ----------------------------------------------------------------------------------------------
\usepackage{custom-style}

%--Beamer Style-----------------------------------------------------------------------------------------------
\usetheme{toninus}








%---------------------------------------------------------------------------------------------------------------------------------------------------
%- D0cum3nt ----------------------------------------------------------------------------------------------------------------------------------
\begin{document}
%------------------------------------------------------------------------------------------------


\subsection{References}

%------------------------------------------------------------------------------------------------
% https://en.wikibooks.org/wiki/LaTeX/Bibliographies_with_biblatex_and_biber
\begin{frame}[t,allowframebreaks]{Extended Bibliography}
	%\nocite{*}
	\bibliographystyle{alpha}
	\bibliography{bibfile}
\end{frame}
%------------------------------------------------------------------------------------------------




%------------------------------------------------------------------------------------------------
\end{document}
%------------------------------------------------------------------------------------------------










%------------------------------------------------------------------------------------------------
\end{document}

