%- HandOut Flag -----------------------------------------------------------------------------------------
\makeatletter
\@ifundefined{ifHandout}{%
  \expandafter\newif\csname ifHandout\endcsname
}{}
\makeatother

%- D0cum3nt ----------------------------------------------------------------------------------------------
\documentclass[beamer,10pt]{standalone}   
%\documentclass[beamer,10pt,handout]{standalone}  \Handouttrue  

\ifHandout
	\setbeameroption{show notes} %print notes   
\fi

	
%- Packages ----------------------------------------------------------------------------------------------
\usepackage{custom-style}
\usepackage{math}




%--Beamer Style-----------------------------------------------------------------------------------------------
\usetheme{toninus}
\usepackage{animate}
\usetikzlibrary{positioning, arrows}
\usetikzlibrary{shapes}

%===========================================================%
\begin{document}
%===========================================================%


%-----------------------------------------------------------%
%-------------------------------------------------------------------------------------------------------------------------------------------------
\begin{frame}[fragile]{Unpacking the title}
\tikzstyle{every picture}+=[remember picture]
	\begin{columns}
    	\begin{column}{.45\textwidth}
    		\onslide<5->{
			\tikz[baseline]{
		            \node[draw=orange!40,anchor=base,text width=5cm] (s1)
		            {\textbf{Goal:}
					is there a morphism 
					\\ \emph{from:} the $L_\infty$-algebra associated with $E^{(n)}$ and the twist $\omega$,
					\\ \emph{to:} the $L_\infty$-algebra associated with $E^{(n+1)}$ without twist?};
			}
		}
	\end{column}
    	\begin{column}{.45\textwidth}
    		\onslide<2->{
			 \tikz[baseline]{
		            \node[draw=blue!40,anchor=base, text width=5cm] (s2)
		            {Higher generalization of Lie algebras where the Jacobi identity is allowed only "up to homotopies".};
			}
		}
	\end{column}
	\end{columns}

	\vfill
	%A canonical morphism between \\ twisted and untwisted higher Courant $L_\infty$-algebras
	\begin{center}
		\large
		A
		 \tikz[baseline]{
		            \node[fill=orange!20,anchor=base] (t1)
		            {Canonical morphism};
			}
		between \\
		 \tikz[baseline]{
		            \node[fill=green!20,anchor=base] (t3)
		            {twisted};
		        } 
		and untwisted 
		 \tikz[baseline]{
	            \node[fill=red!20,anchor=base] (t4)
	            {higher Courant};
		}
		 \tikz[baseline]{
	            \node[fill=blue!20,anchor=base] (t2)
	            {$L_\infty$-algebras};
		}		
	\end{center}

	\vfill

	\begin{columns}
    	\begin{column}{.45\textwidth}
    		\onslide<4->{
	 		 \tikz[baseline]{
	            \node[draw=green!40,anchor=base,text width=5cm] (s3)
	            {any closed differential form $$\omega\in\Omega^{n+2}(M)$$ can be use to \emph{twist} the binary bracket on $\Gamma(E^{(n)})$.};
	         }
		}		   	
		\end{column}
    	\begin{column}{.45\textwidth}
    		\onslide<3->{
				\tikz[baseline]{
	            \node[draw=red!40,anchor=base,text width=5cm] (s4)
	            {\textbf{Higher Courant algebroids}\\
					smooth vector bundles
				$$ E^{(n)}= TM \oplus \wedge^{n}T^\ast M \to M$$

				 	the space of global sections $\Gamma(E^{(n)})$ forms an $L_\infty$-algebra!
				};
	           }	
			}
		\end{column}
	\end{columns}

	\begin{tikzpicture}[overlay]
        \path[->,draw=orange!40, line width=.5mm]<5-> (s1) edge [bend right] (t1);
        \path[->,draw=blue!40, line width=.5mm]<2-> (s2) edge [bend left] (t2);
        \path[->,draw=green!40, line width=.5mm]<4-> (s3) edge [bend left] (t3);
        \path[->,draw=red!40, line width=.5mm]<3-> (s4) edge [bend right] (t4);
	\end{tikzpicture}


\end{frame}
\note[itemize]{
	\item we deal with the category of \emph{L_\infty-algebras}. These are a generalziation of dg-Lie algebras, where we ask the Jacobi identity to be satisfied up to exact terms. The corresponding primitives are given by certain higher brackets.
	\item the (standard) higher Courant algebroid is part of a family of vector bundle over $M$. In degree 0 we have the standard Lie algebroid, in degree 1 the standard Courant algebroid an we can go higher in this sequence.
	\item 
	\item Conventions:
	\\- $M$ and $G$ are connected,
	\\- actions $\theta:G \curvearrowright M$ are always smooth
	\\- $\xi,\eta\in\mathfrak{g}$,
	\\- for $\mu\in\Omega^*(M,\mathfrak{g}^*)$ and $\xi\in\mathfrak{g}$, write
			\[
				\mu_\xi := \langle\mu,\xi\rangle \;{\color{black!50}\in\Omega^*(M)}
			\]
			for the ``$\xi$th component'' of $\mu$.
}
%-----------------------------------------------------------%

%-----------------------------------------------------------%
\outline
%-----------------------------------------------------------%

%-----------------------------------------------------------%
\subsection{Symplectic Case}
%-----------------------------------------------------------%
\begin{frame}{Symplectic Case}
	\begin{itemize}
		\item \textbf{Goal:} Find a canonical morphism between the twisted and untwisted higher Courant $L_\infty$-algebras.
		\item \textbf{Strategy:} Use the cone construction to obtain a homotopy fiber of the morphism.
		\item \textbf{Result:} The morphism is given by a certain 2-plectic structure on the cone.
	\end{itemize}
\end{frame}
\note[itemize]{
	\item 
}%-----------------------------------------------------------%
\subsection{2-plectic Case}
%-----------------------------------------------------------%
\begin{frame}{2-plectic Case}
	\begin{itemize}
		\item \textbf{Goal:} Extend the result to the 2-plectic case.
		\item \textbf{Strategy:} Use the cone construction to obtain a homotopy fiber of the morphism.
		\item \textbf{Result:} The morphism is given by a certain 3-plectic structure on the cone.
	\end{itemize}
\end{frame}
\note[itemize]{
	\item 
}
%-----------------------------------------------------------%
\subsection{Goals}
%-----------------------------------------------------------%
\begin{frame}{Goals}
	\begin{itemize}
		\item \textbf{Goal:} Find a canonical morphism between the twisted and untwisted higher Courant $L_\infty$-algebras.
		\item \textbf{Strategy:} Use the cone construction to obtain a homotopy fiber of the morphism.
		\item \textbf{Result:} The morphism is given by a certain $(n+1)$-plectic structure on the cone.
	\end{itemize}
\end{frame}
\note[itemize]{
	\item 
}



%-----------------------------------------------------------%
\ifstandalone
% https://en.wikibooks.org/wiki/LaTeX/Bibliographies_with_biblatex_and_biber
\begin{frame}[t,allowframebreaks]{Partial Bibliography}
	\nocite{Miti2021}
	\bibliographystyle{alpha}
	\bibliography{bibfile}
\end{frame}
\fi
%-----------------------------------------------------------%

\end{document}