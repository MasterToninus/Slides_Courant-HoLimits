%- HandOut Flag -----------------------------------------------------------------------------------------
\makeatletter
\@ifundefined{ifHandout}{%
  \expandafter\newif\csname ifHandout\endcsname
}{}
\makeatother

%- D0cum3nt ----------------------------------------------------------------------------------------------
\documentclass[beamer,10pt]{standalone}   
%\documentclass[beamer,10pt,handout]{standalone}  \Handouttrue  

\ifHandout
	\setbeameroption{show notes} %print notes   
\fi

	
%- Packages ----------------------------------------------------------------------------------------------
\usepackage{custom-style}
\usepackage{math}




%--Beamer Style-----------------------------------------------------------------------------------------------
\usetheme{toninus}
\usepackage{animate}
\usetikzlibrary{positioning, arrows}
\usetikzlibrary{shapes}

%===========================================================%
\begin{document}
%===========================================================%


%-----------------------------------------------------------%
%-------------------------------------------------------------------------------------------------------------------------------------------------
\begin{frame}[fragile]{Keywords}
\tikzstyle{every picture}+=[remember picture]
	\begin{columns}
    	\begin{column}{.45\textwidth}
    		\onslide<5->{
			\tikz[baseline]{
		            \node[draw=orange!40,anchor=base,text width=5cm] (s1)
		            { Un canonical mordfismo?};
			}
		}
	\end{column}
    	\begin{column}{.45\textwidth}
    		\onslide<2->{
			 \tikz[baseline]{
		            \node[draw=blue!40,anchor=base, text width=5cm] (s2)
		            {Higher generalization of Lie algebras where the Jacobi identity is allowed only "up to homotopies".};
			}
		}
	\end{column}
	\end{columns}

	\vfill
	%A canonical morphism between \\ twisted and untwisted higher Courant $L_\infty$-algebras
	\begin{center}
		\large
		A
		 \tikz[baseline]{
		            \node[fill=orange!20,anchor=base] (t1)
		            {Canonical morphism};
			}
		between \\
		 \tikz[baseline]{
		            \node[fill=green!20,anchor=base] (t3)
		            {twisted};
		        } 
		and untwisted 
		 \tikz[baseline]{
	            \node[fill=red!20,anchor=base] (t4)
	            {higher Courant};
		}
		 \tikz[baseline]{
	            \node[fill=blue!20,anchor=base] (t2)
	            {$L_\infty$-algebras};
		}		
	\end{center}

	\vfill

	\begin{columns}
    	\begin{column}{.45\textwidth}
    		\onslide<4->{
	 		 \tikz[baseline]{
	            \node[draw=green!40,anchor=base,text width=5cm] (s3)
	            {Twists};
	         }
		}		   	
		\end{column}
    	\begin{column}{.45\textwidth}
    		\onslide<3->{
				\tikz[baseline]{
	            \node[draw=red!40,anchor=base,text width=5cm] (s4)
	            {Courant algoids};
	           }	
			}
		\end{column}
	\end{columns}

	\begin{tikzpicture}[overlay]
        \path[->,draw=orange!40]<5-> (s1) edge [bend right] (t1);
        \path[->,draw=blue!40]<2-> (s2) edge [bend left] (t2);
        \path[->,draw=green!40]<4-> (s3) edge [bend left] (t3);
        \path[->,draw=red!40]<3-> (s4) edge [bend right] (t4);
	\end{tikzpicture}


\end{frame}
\note[itemize]{
	\item Provide a vague idea of the words that make up the title.
	\item
	\item Conventions:
	\\- $M$ and $G$ are connected,
	\\- actions $\theta:G \curvearrowright M$ are always smooth
	\\- $\xi,\eta\in\mathfrak{g}$,
	\\- for $\mu\in\Omega^*(M,\mathfrak{g}^*)$ and $\xi\in\mathfrak{g}$, write
			\[
				\mu_\xi := \langle\mu,\xi\rangle \;{\color{black!50}\in\Omega^*(M)}
			\]
			for the ``$\xi$th component'' of $\mu$.
}
%-----------------------------------------------------------%

%-----------------------------------------------------------%
\outline
%-----------------------------------------------------------%

%-----------------------------------------------------------%
\subsection{Symplectic Case}
%-----------------------------------------------------------%
\begin{frame}{Symplectic Case}
	\begin{itemize}
		\item \textbf{Goal:} Find a canonical morphism between the twisted and untwisted higher Courant $L_\infty$-algebras.
		\item \textbf{Strategy:} Use the cone construction to obtain a homotopy fiber of the morphism.
		\item \textbf{Result:} The morphism is given by a certain 2-plectic structure on the cone.
	\end{itemize}
\end{frame}
\note[itemize]{
	\item 
}%-----------------------------------------------------------%
\subsection{2-plectic Case}
%-----------------------------------------------------------%
\begin{frame}{2-plectic Case}
	\begin{itemize}
		\item \textbf{Goal:} Extend the result to the 2-plectic case.
		\item \textbf{Strategy:} Use the cone construction to obtain a homotopy fiber of the morphism.
		\item \textbf{Result:} The morphism is given by a certain 3-plectic structure on the cone.
	\end{itemize}
\end{frame}
\note[itemize]{
	\item 
}
%-----------------------------------------------------------%
\subsection{Goals}
%-----------------------------------------------------------%
\begin{frame}{Goals}
	\begin{itemize}
		\item \textbf{Goal:} Find a canonical morphism between the twisted and untwisted higher Courant $L_\infty$-algebras.
		\item \textbf{Strategy:} Use the cone construction to obtain a homotopy fiber of the morphism.
		\item \textbf{Result:} The morphism is given by a certain $(n+1)$-plectic structure on the cone.
	\end{itemize}
\end{frame}
\note[itemize]{
	\item 
}



%-----------------------------------------------------------%
\ifstandalone
% https://en.wikibooks.org/wiki/LaTeX/Bibliographies_with_biblatex_and_biber
\begin{frame}[t,allowframebreaks]{Partial Bibliography}
	\nocite{Miti2021}
	\bibliographystyle{alpha}
	\bibliography{bibfile}
\end{frame}
\fi
%-----------------------------------------------------------%

\end{document}