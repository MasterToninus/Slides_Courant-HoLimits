%- HandOut Flag -----------------------------------------------------------------------------------------
\makeatletter
\@ifundefined{ifHandout}{%
  \expandafter\newif\csname ifHandout\endcsname
}{}
\makeatother

%- D0cum3nt ----------------------------------------------------------------------------------------------
\documentclass[beamer,10pt]{standalone}   
%\documentclass[beamer,10pt,handout]{standalone}  \Handouttrue  

\ifHandout
	\setbeameroption{show notes} %print notes   
\fi

	
%- Packages ----------------------------------------------------------------------------------------------
\usepackage{custom-style}
\usepackage{math}




%--Beamer Style-----------------------------------------------------------------------------------------------
\usetheme{toninus}
\usepackage{animate}
\usetikzlibrary{positioning, arrows}
\usetikzlibrary{shapes}

%===========================================================%
\begin{document}
%===========================================================%


%-----------------------------------------------------------%
%-------------------------------------------------------------------------------------------------------------------------------------------------
\begin{frame}[fragile]{Unpacking the title}
\tikzstyle{every picture}+=[remember picture]
	\begin{columns}
    	\begin{column}{.45\textwidth}
    		\onslide<5->{
			\tikz[baseline]{
		            \node[draw=orange!40,anchor=base,text width=5cm] (s1)
		            {\textbf{Goal:}
					is there a morphism 
					\\ \emph{from:} the $L_\infty$-algebra associated with $E^{(r)}$ and the twist $\sigma$,
					\\ \emph{to:} the $L_\infty$-algebra associated with $E^{(r+1)}$ without twist?};
			}
		}
	\end{column}
    	\begin{column}{.45\textwidth}
    		\onslide<2->{
			 \tikz[baseline]{
		            \node[draw=blue!40,anchor=base, text width=5cm] (s2)
		            {Higher generalization of Lie algebras where the Jacobi identity is allowed only "up to homotopies".};
			}
		}
	\end{column}
	\end{columns}

	\vfill
	%A canonical morphism between \\ twisted and untwisted higher Courant $L_\infty$-algebras
	\begin{center}
		\large
		A
		 \tikz[baseline]{
		            \node[fill=orange!20,anchor=base] (t1)
		            {Canonical morphism};
			}
		between \\
		 \tikz[baseline]{
		            \node[fill=green!20,anchor=base] (t3)
		            {twisted};
		        } 
		and untwisted 
		 \tikz[baseline]{
	            \node[fill=red!20,anchor=base] (t4)
	            {higher Courant};
		}
		 \tikz[baseline]{
	            \node[fill=blue!20,anchor=base] (t2)
	            {$L_\infty$-algebras};
		}		
	\end{center}

	\vfill

	\begin{columns}
    	\begin{column}{.45\textwidth}
    		\onslide<4->{
	 		 \tikz[baseline]{
	            \node[draw=green!40,anchor=base,text width=5cm] (s3)
	            {any closed differential form $$\sigma\in\Omega^{r+2}(M)$$ can be use to \emph{twist} the binary bracket on $\Gamma(E^{(r)})$.};
	         }
		}		   	
		\end{column}
    	\begin{column}{.45\textwidth}
    		\onslide<3->{
				\tikz[baseline]{
	            \node[draw=red!40,anchor=base,text width=5cm] (s4)
	            {\textbf{Higher Courant algebroids}\\
					smooth vector bundles
				$$ E^{(r)}= TM \oplus \wedge^{r}T^\ast M \to M$$
				\medskip
				 	s.t. the space of global sections $\Gamma(E^{(r)})$ forms an $L_\infty$-algebra!
				};
	           }	
			}
		\end{column}
	\end{columns}

	\begin{tikzpicture}[overlay]
        \path[->,draw=orange!40, line width=.5mm]<5-> (s1) edge [bend right] (t1);
        \path[->,draw=blue!40, line width=.5mm]<2-> (s2) edge [bend left] (t2);
        \path[->,draw=green!40, line width=.5mm]<4-> (s3) edge [bend left] (t3);
        \path[->,draw=red!40, line width=.5mm]<3-> (s4) edge [bend right] (t4);
	\end{tikzpicture}


\end{frame}
\note[itemize]{
	\item we deal with the category of \emph{$L_\infty$-algebras}. These are a generalziation of dg-Lie algebras, where we ask the Jacobi identity to be satisfied up to exact terms. The corresponding primitives are given by certain higher brackets.
	\item the (standard) higher Courant algebroid is part of a family of vector bundle over $M$. In degree 0 we have the standard Lie algebroid, in degree 1 the standard Courant algebroid an we can go higher in this sequence.
	\item 
	\item Conventions:
	\\- $M$ and $G$ are connected,
	\\- actions $\theta:G \curvearrowright M$ are always smooth
	\\- $\xi,\eta\in\mathfrak{g}$,
	\\- for $\mu\in\Omega^*(M,\mathfrak{g}^*)$ and $\xi\in\mathfrak{g}$, write
			\[
				\mu_\xi := \langle\mu,\xi\rangle \;{\color{black!50}\in\Omega^*(M)}
			\]
			for the ``$\xi$th component'' of $\mu$.
}
%-----------------------------------------------------------%

%-----------------------------------------------------------%
\outline
\checkpoint
%-----------------------------------------------------------%

%-----------------------------------------------------------%
\subsection{Symplectic Case}
%-----------------------------------------------------------%
\begin{frame}{(Pre)-symplectic Case}
	$M$ manifold; $\sigma \in \Omega^2_{\mathrm{cl}}$ closed 2-form;
	\footnote{\color{gray}We omit to write $(M)$ when considering differential forms $\Omega(M)$ and vector fields $\X(M)$ on $M$.}
	\vfill\pause

	\begin{defblock}[Hamiltonian vector fields $\X_{\sigma,\ham} \subseteq \X$]
		\begin{displaymath}
			\X_{\sigma,\ham} = \{ X \in \X \mid \iota_X \sigma ~~\textrm{is exact}\}
		\end{displaymath}
		\blfootnote{i.e. $\exists f \in \Omega^0$ s.t. $\iota_X \sigma + \d f=0$}
	\end{defblock}
	\vfill\pause

	\begin{defblock}[Hamiltonian pairs ]
		\begin{displaymath}
			\Ham^{(0)}_\sigma = \{(X,f) \mid \iota_X \sigma + \d f = 0 \}
	\subseteq \X\oplus \Omega^0 = \Gamma(E^{(0)})
		\end{displaymath}
	\end{defblock}
	\vfill\pause

	\begin{defblock}[Standard Lie algebroid]
		Vector bundle
		\begin{displaymath}
			E^{(0)} = TM \oplus \R = TM \oplus \wedge^0 T^* M \to M
		\end{displaymath}
		... with a \emph{binary bracket} on sections and a \emph{anchor map} \small(see below)\normalsize.
	\end{defblock}
\end{frame}
\note[itemize]{
	\item Ok, we mentioned a morphism. How does it look like in the lower case?
	\item The story starts from presymplectic geometry.
	\item out of the data $\sigma$ we define a subspace of vector fields. (actually a Lie subalgebra since $\iota_{[x,y]}=\Lie_x \iota_y - \iota_y \Lie_x$).
	\item We don't need non degeneracy to have a Lie algebra, we just say that exists a primitive function without specifying it.
	\item An improved notion is to consider Hamiltonian pairs where elements are a specification of a v.field together with an hamiltonian function.
	\item the index $(0)$ appearing is to recall the degree of hamiltonian forms and suggests that we can go higher -> we can devise a pattern.
	\item Hamiltonian forms can be seen as elements of the space of sections of certain vector bundle $E^{(0)}$.
	\item Again index $(0)$ is to devise a pattern going toward courant algebroids and their higher generalizations.
}%-----------------------------------------------------------%

%-----------------------------------------------------------%
\begin{frame}{(Pre)-symplectic Case: algebraic structures (1)}
	\begin{lemblock}[$\Ham^{(0)}_\sigma$ carries a Lie algebra structure:]
		\begin{equation}\label{eq:Bracket-HamPairs}
		\big\{ (X,f), (Y,g)\big\} = \big( [X,Y], \sigma(X,Y)\big)
		\end{equation}
	\blfootnote{Note: $\sigma(X,Y)$ is an Hamiltonian 1-form for $[X,Y]$!}
	\end{lemblock}
	\vfill\pause

	\begin{remblock}["Observables" Poisson algebra]
		Cleary $\Ham^{(0)}_\sigma \twoheadrightarrow \X_{\sigma,\ham}$ \small (forgetting the chosen primitive function) \normalsize
		\\
		if $\sigma$ is non-degenerate, then $\Ham^{(0)}_\sigma \cong \Omega^0$ and \eqref{eq:Bracket-HamPairs} is the usual \emph{Poisson bracket}.
		\footnote{Note: $\sigma(X,Y) = \iota_Y \iota_X \sigma = -\iota_Y \d f = \iota_X \d g \qquad \forall(X,f), (Y,g) \in \Ham^{(0)}_\sigma$.}
	\end{remblock}	 
	\vfill\pause
 
	\begin{lemblock}[$\Gamma(E^{(0)}) = \X\oplus \Omega^0$ carries a Lie algebra structure:]
		\begin{equation}\label{eq:Bracket-LieAlgoid}
			\big\{ (X,f), (Y,g)\big\} = \big( [X,Y],~ \Lie_X g - \Lie_Y f + \sigma(X,Y) \big)
		\end{equation}
	\end{lemblock}
\end{frame}
\note[itemize]{
	\item 
}%------------------------------------------------------------%

%-----------------------------------------------------------%
\begin{frame}{(Pre)-symplectic Case: algebraic structures (2)}	
	\begin{notblock}
		\begin{itemize}
			\item[$\bullet$] $\Courant[\sigma]{0} := \X\oplus \Omega^{0}$ with its Lie algebra structure \eqref{eq:Bracket-LieAlgoid}
			\item[$\bullet$] $\Rogers[\sigma]{0} := \Ham^{(0)}_\sigma$ with its Lie algebra structure \eqref{eq:Bracket-HamPairs}
		\end{itemize}
		%
		\begin{displaymath}
			\begin{tikzcd}[ampersand replacement=\&]
				\Rogers[\sigma]{0}   \ar[r,two heads]\ar[d,hook]\& \X_{\sigma,\ham}
				\\ 
				\Courant[\sigma]{0}
			\end{tikzcd}
		\end{displaymath}
	\end{notblock}
	\vfill\pause


	\seprule
	\begin{upshotblock}
		\begin{itemize}
			\item this was the $r=0$ case :
				\begin{itemize}
					\item[$\bullet$] $\sigma$ is a degree $r+2$ closed form,
					\item[$\bullet$] $\Rogers[\sigma]{r},\Courant[\sigma]{r}$ are Lie algebras.
				\end{itemize}
				\pause\medskip
			\item Let us now move from $r=0$ to $r=1$.
		\end{itemize}
	\end{upshotblock}

\end{frame}
\note[itemize]{
	\item 
}%------------------------------------------------------------%	
	\subsection{$r=1$: the 2-plectic Case}
%-----------------------------------------------------------%

%-----------------------------------------------------------%
\begin{frame}{2-plectic case}
	Fix a closed 3-form $\sigma \in \Omega^3_{\mathrm{cl}}$;
	\vfill\pause

	\begin{defblock}[Hamiltonian vector fields  \emph{(same as before!)}]
		\begin{displaymath}
			\X_{\sigma,\ham} = \{ X \in \X \mid \iota_X \sigma ~~\textrm{is exact}\}
		\end{displaymath}
	\end{defblock}
	\vfill\pause

	\begin{defblock}[Hamiltonian pairs \emph{(same as before!)}]
		\begin{displaymath}
			\Ham^{(1)}_\sigma = \{(X,\omega) \mid \iota_X \sigma + \d f = 0 \}
	\subseteq \X\oplus \Omega^1 = \Gamma(E^{(1)})
		\end{displaymath}
	\end{defblock}
	\vfill\pause

	\begin{defblock}[Standard Courant algebroid]
		Vector bundle
		\begin{displaymath}
			E^{(1)} = TM \oplus  T^* M \to M
		\end{displaymath}
		\pause
		... with a \emph{binary bracket} on sections and a \emph{anchor map} \small(see below)\normalsize.
	\end{defblock}	

\end{frame}
\note[itemize]
{
	\item .
}
%-----------------------------------------------------------%

%-----------------------------------------------------------%
\begin{frame}[fragile]{2-plectic case: algebraic structures}
	\begin{itemize}
		\item \emph{do $\Ham^{(1)}_\sigma$ and $E^{(1)}$ carry a Lie algebra structure?}\pause
			\begin{itemize}
				\item[$\bullet$] $\Gamma(E^{(1)})$ has a canonical antisymmetric bracket (called \emph{Courant bracket});\pause
				\item[$\bullet$] Same for $\Ham^{(1)}_\sigma$ (generalizing \eqref{eq:Bracket-HamPairs} to 1-forms).\pause
			\end{itemize}
		\item They do \underline{not} satisfy the Jacobi identity on the nose but only \emph{"up to homotopy"}.\pause
		 $\Rightarrow$ \underline{there could be a $L_\infty$-algebra lying around!}
	\end{itemize}	
	\vfill

	%
	\begin{propblock}[(Rogers 2013)]
		\vspace{-2em}
		\begin{displaymath}
			\begin{tikzcd}[ampersand replacement=\&]
				\& \text{\small (deg $-1$)} \& \text{\small (deg $0$)} \\[-2.5em]
				\Rogers[\sigma]{1}   \ar[r,phantom,":="]
				\&
				\Omega^0 \ar[r,"{(0,\d)}"] \ar[d,"id"]
				\& \Ham_{\sigma}^{(1)} \ar[d,hook]
				\&[.7em] \parbox{9em}{\small carries a distinguished $L_\infty$-algebra structure \cite{Rogers2010}}
				\\
				\Courant[\sigma]{1} \ar[r,phantom,":="]
				\& \Omega^0 \ar[r,"{(0,\d)}"]
				\& \Gamma(E^{(1)}) \&
				\parbox{9em}{\small forms a $L_2$-algebra with the courant bracket \cite{Roytenberg1998}}
			\end{tikzcd}
		\end{displaymath}


	Vertical arrows are the linear part of a $L_\infty$-morphism $$ \Rogers[\sigma]{1} \to \Courant[\sigma]{1} ~.$$
	\end{propblock}
\end{frame}
\note[itemize]
{
	\item .
}
%-----------------------------------------------------------%





%-----------------------------------------------------------%
\subsection{$r$-plectic Case}
%-----------------------------------------------------------%
%-----------------------------------------------------------%
\begin{frame}{Going higher!}
	%
	Fix a closed $(r$+$1)$-form $\sigma \in \Omega^{r+1}_{\mathrm{cl}}$~;
	\vfill\pause

	\begin{defblock}[Hamiltonian vector fields  \emph{(same as before!)}]
		\small \color{gray}
		\begin{displaymath}
			\X_{\sigma,\ham} = \{ X \in \X \mid \iota_X \sigma ~~\textrm{is exact}\}
		\end{displaymath}
	\end{defblock}
	\vfill\pause

	\begin{defblock}[Hamiltonian pairs \emph{(same as before!)}]
		\small \color{gray}
		\begin{displaymath}
			\Ham^{(r-1)}_\sigma = 
		\{ (X,\omega) \mid \iota_X \sigma + \d \omega  = 0 \}
		\subseteq \X\oplus \Omega^{r-1} = \Gamma(E^{(r-1)})
		\end{displaymath}
	\end{defblock}
	\vfill\pause

	\begin{defblock}[Higher Courant (Vinogradov) algebroid ]
		Vector bundle
		\begin{displaymath}
			E^{(r-1)} = TM \oplus \wedge^{r-1} T^* M \to M
		\end{displaymath}
		... with a \emph{binary bracket} on sections and a \emph{anchor map} \small(see below)\normalsize.
	\end{defblock}		
	\vfill

\end{frame}
\note[itemize]{
	\item 
}
%-----------------------------------------------------------%


%-----------------------------------------------------------%
\begin{frame}{Going higher!~Algebraic structures}
	%
	\begin{propblock}
		\vspace{-1em}
		\begin{displaymath}
		\begin{tikzcd}[ampersand replacement=\&,column sep = small]
			\& \text{\small (deg $1-r$)} \& \cdots \& \&\text{\small (deg $-1$)} \& \text{\small (deg $0$)} \\[-.5em]
			\Rogers[\sigma]{r-1}   \ar[r,phantom,":="]
			\&
			\Omega^0 \ar[r,"{\d}"] \ar[d,"id"]
			\&
			\Omega^1 \ar[r,"{\d}"] \ar[d,"id"]
			\&
			\cdots
			\&
			\Omega^{r-2} \ar[r,"{(0,\d)}"] \ar[d,"id"]
			\& \Ham_{\sigma}^{(r-1)} \ar[d,hook]
			\\
			\Courant[\sigma]{r-1} \ar[r,phantom,":="]
			\&
			\Omega^0 \ar[r,"{\d}"]
			\&
			\Omega^1 \ar[r,"{\d}"] 
			\&
			\cdots
			\&
			\Omega^{r-2} \ar[r,"{(0,\d)}"]
			\& \Gamma(E^{(r-1)})
		\end{tikzcd}
	\end{displaymath}
	\vspace{2em}

	\begin{itemize}
		\item \cite{Rogers2010} endows $\Rogers[\sigma]{r-1}$ with a $L_\infty$-algebra structure.\pause
		\item \cite{Zambon2012} endows $\Courant[\sigma]{r-1}$ with a $L_\infty$-algebra structure.\pause
		\item The vertical arrows give a morphism of cochain complexes.\pause
		\item \cite{Miti2024} shows that the vertical arrows are the linear part of a $L_\infty$-morphism
			$$\Rogers[\sigma]{r-1} \to \Courant[\sigma]{r-1}~.$$
	\end{itemize}

	\end{propblock}

\end{frame}
\note[itemize]{
	\item 
}
%-----------------------------------------------------------%

%-----------------------------------------------------------%
\subsection{Goals}
%-----------------------------------------------------------%

%-----------------------------------------------------------%
\begin{frame}{Goals}
	\begin{itemize}
		\item \cite{Miti2024} explicitly contructs a morphism of $L_\infty$-algebras
		\begin{displaymath}
			\begin{tikzcd}[ampersand replacement=\&]
				\Rogers[\sigma]{r-1} \ar[r] \& 
				\Courant[\sigma]{r-1} \ar[r,phantom] \&
				\phantom{\Courant[]{r}}
			\end{tikzcd}
		\end{displaymath}
	\end{itemize}
	\vfill

	\begin{claimblock}
		\centering We can prove that we can go on:
		\vspace{-.5em}
		\begin{displaymath}
			\begin{tikzcd}[ampersand replacement=\&]
				\Rogers[\sigma]{r-1} \ar[r] \& 
				\Courant[\sigma]{r-1} \ar[r] \&
				\Courant[]{r}
			\end{tikzcd}
		\end{displaymath}
	\end{claimblock}
	\vfill

	\begin{claimblock}
		\centering
		We rely on an algebraic framework that explains the \emph{existence} and \emph{\underline{naturality}} of the morphism introduced in \cite{Miti2024}.
	\end{claimblock}
	\vfill

\end{frame}
\note[itemize]{
	\item in princple one would have $$ \dots \to \Courant[\sigma]{r-1} \to \Courant[]{r} \to \Courant[]{r+1}\to \dots$$
	however the image of the last two arrows is basically $\Cartan$.
}
%-----------------------------------------------------------%


%-----------------------------------------------------------%
\ifstandalone
% https://en.wikibooks.org/wiki/LaTeX/Bibliographies_with_biblatex_and_biber
\begin{frame}[t,allowframebreaks]{Partial Bibliography}
	\nocite{Miti2021}
	\bibliographystyle{alpha}
	\bibliography{bibfile}
\end{frame}
\fi
%-----------------------------------------------------------%

\end{document}