%+----------------------------------------------------------------------------+
%| SLIDES: 
%| Chapter: Complementary material - details on eventual questions
%| Author: Antonio miti
%| Event: PHD preliminary Defence
%+----------------------------------------------------------------------------+

%- HandOut Flag -----------------------------------------------------------------------------------------
\newif\ifHandout

%- D0cum3nt ----------------------------------------------------------------------------------------------
\documentclass[beamer,10pt]{standalone}   
%\documentclass[beamer,10pt,handout]{standalone}  \Handouttrue  

%- HandOut Flag -----------------------------------------------------------------------------------------
\ifHandout
	\setbeameroption{show notes} %print notes   
\fi

	
%- Packages ----------------------------------------------------------------------------------------------
\usepackage{custom-style}
\usepackage{math}

%--Beamer Style-----------------------------------------------------------------------------------------------
\usetheme{toninus}



\providecommand{\blank}{\text{\textvisiblespace}}


\newcommand{\subsectiontitle}{
  \begin{frame}
  \vfill
  \centering
  \begin{beamercolorbox}[sep=8pt,center,shadow=true,rounded=true]{title}
    \usebeamerfont{title}\insertsectionhead\par%
    \usebeamerfont{title}\insertsubsectionhead\par%
  \end{beamercolorbox}
  \vfill
  \end{frame}
}

\providecommand{\blank}{\text{\textvisiblespace}}




%===========================================================%

%- D0cum3nt %===========================================================%

\begin{document}
%===========================================================%


%##################################################################################
\begin{frame}
	\begin{center}
	\Huge\emph{Supplementary Material}
	\end{center}
\end{frame}
\note[itemize]{
	\item
}
\addtocounter{framenumber}{-1}
%##################################################################################





%===================================================================================
\section{Background}
%===================================================================================



%-------------------------------------------------------------------------------------------------------------------------------------------------
\subsection{Symplectic Manifolds}
%-------------------------------------------------------------------------------------------------------------------------------------------------
 
%-------------------------------------------------------------------------------------------------------------------------------------------------
\begin{frame}[fragile]{Lie $\infty$-algebra of Observables (higher observables) }
	Let be $(M,\omega)$ a $n$-plectic manifold.
	  	\vfill
	\begin{defblock}[$L_\infty$-algebra of observables ~\emph{(Rogers)} \cite{Rogers2010}]
		\medskip
		\hspace{.25em} Is a cochain-complex $(L,\{\cdot\}_1)$ \\
		\vspace{-1em}
		\begin{center}
			\includestandalone[width=0.95\textwidth]{Pictures/Frame_Observables}
		\end{center}
		\onslide<2->{
			\bigskip
			\hspace{.25em} with $n$ (skew-symmetric) multibrackets $(2 \leq k \leq n+1)$\\
			\vspace{-1em}
			\onslide<3->{
				\begin{center}
					\includestandalone{Pictures/Equation_Multibracket}	
				\end{center}
			}
			\medskip
		}
		%
	\end{defblock}
%	\onslide<3->{
%		\emph{Higher analogue} of the \emph{Poisson algebra structure} associated to a symplectic mfd.
%	\vfill
%	\begin{columns}
%		\hfill
%		\begin{column}{.11\linewidth}	
%			If $n>1$:
%			
%		\end{column}	
%		\begin{column}{.8\linewidth}
%		\begin{itemize}
%			\item[\xmark] \textcolor{red}{we lose} :\quad multiplication of observables, Jacobi equation;
%			%\\ \hspace*{4.25em} full-fledged Jacobi equation;
%			\item[\cmark] \textcolor{green}{we gain} :\quad brackets with arities different than two,\\
%			\hspace*{4.25em}
%			 Jacobi equation \emph{up to homotopies}.
%		\end{itemize}		
%		\end{column}		
%	\end{columns}
%	}
  \end{frame}
%-------------------------------------------------------------------------------------------------------------------------------------------------

%-------------------------------------------------------------------------------------------------------------------------------------------------
\begin{frame}[t, fragile]{Research Framework:  \textbf{multisymplectic geometry}} %Fragile -->workaround tikzcd
	\begin{center}
		$-$ \emph{multisymplectic means \textbf{going higher} in the degree of $\omega$} $-$
	\end{center}
	\pause
	\begin{defblock}[$n$-plectic manifold ~\emph{(Cantrijn, Ibort, De Le\'on)} \cite{Cantrun2017}]
		\includestandalone[width=0.95\textwidth]{./Pictures/Figure_multisym}	
	\end{defblock}
	%
	\vfill
	%
	%
	\pause
	\begin{block}{Examples:}
		\begin{itemize}
			\item[$\bullet$] 1-plectic $=$ symplectic
			\item[$\bullet$] Any oriented $(n+1)$-dimensional manifold is $n$-plectic w.r.t. the volume form.
			\item[$\bullet$] The multicotangent bundle $\Lambda^n T^\ast Q$ is naturally $n$-plectic.
		\end{itemize}
	\end{block}			 
%
	\pause
	\begin{block}{Historical motivation}
		Mechanics: geometrical foundations of \textit{(first-order)} field theories.
		\begin{itemize}
		 \item[•] Kijowski, W. Tulczyjew \cite{Kijowski1979}; %(1979)
		 \item[•] Cariñena, Crampin, Ibort \cite{Carinena1991b};% (1991)
		 \item[•] Gotay, Isenberg, Marsden, Montgomery \cite{Gimmsy1};%(1998)
		 \\ $\cdots$
		\end{itemize}
	\end{block}
\end{frame}
%-------------------------------------------------------------------------------------------------------------------------------------------------


%-------------------------------------------------------------------------------------------------------------------------------------------------
\begin{frame}{Observables in \textbf{multisymplectic geometry}}
	%
	\begin{defblock}[Hamiltonian $(n-1)$-forms]
		\begin{displaymath}
			\Omega^{n-1}_{ham}(M,\omega) 	:=
			\biggr\{ \sigma \in  \Omega^{n-1}(M) \; \biggr\vert \; 
				\exists \vHam_\sigma \in \mathfrak{X}(M) ~:~ 
				\tikz[baseline,remember picture]{\node[rounded corners,
                        fill=orange!5,draw=orange!30,anchor=base]            
            			(target) {$d \sigma = -\iota_{\vHam_\sigma} \omega$ };
            	}				
				~\biggr\} 
			\end{displaymath}
	\end{defblock}
	%
	\onslide<2>{
		\tikz[overlay,remember picture]
		{
			\node[rounded corners,
                 fill=orange!5,draw=orange!30,anchor=base]
            	 (base) at ($(current page.north east)-(2,1)$) [rotate=-0,text width=3.5cm,align=center] {\footnotesize{\textcolor{red}{Hamilton-DeDonder-Weyl \\equation}}};
		}	
		\begin{tikzpicture}[overlay,remember picture]
		    	\path[->] (base.south east) edge[bend left,red](target.east);
	    \end{tikzpicture}
	}
	%
	\vspace{-1em}
	\pause
	\begin{columns}[T]
		\setlength{\belowdisplayskip}{5pt}
		\begin{column}{.50\linewidth}
			%
			\centering \it
			$-$ symplectic case $-$
			\onslide<3->{
			\begin{thmblock}[Observables Poisson algebra]
				$C^\infty(M,\omega)$ endowed with
				\vspace{-.5em}
				\begin{displaymath}
					\lbrace \sigma_1, \sigma_2 \rbrace =			
					~ - \iota_{\vHam_1}\iota_{\vHam_2} \omega 
					~= \mathcal{L}_{\vHam_1} \sigma_2
				\end{displaymath}			
				forms a Poisson algebra.
			\end{thmblock}
			}
			%
			\onslide<4->{
			\vspace{1em}
			\begin{itemize}
				\item[\cmark] Skew-symmetric;
				\item[\cmark] multiplication of observables;
				\item[\cmark] Leibniz Rule;
				\item[\cmark] Jacobi equation;
			\end{itemize}		
			}		
		\end{column}	
		%
		\onslide<1->{\vrule{}}
		%
		\begin{column}{.50\linewidth}
			\centering \it
			$-$ $n$-plectic case $-$
			\onslide<5->{			
			\begin{thmblock}[Observables $L_\infty$-algebra]
				$\Omega^{n-1}_{ham}(M,\omega)$ endowed with
				\vspace{-.5em}
				\begin{displaymath}
					\lbrace \sigma_1, \sigma_2 \rbrace =			
					~ - \iota_{\vHam_1}\iota_{\vHam_2} \omega 
				\end{displaymath}			
				can be extended to a \\ $L_\infty-algebra$.
			\end{thmblock}
			}
			%
			\onslide<6->{
			\begin{itemize}
				\item[\cmark] Skew-symmetric;
				\item[\xmark] multiplication of observables;
				\item[\xmark] Jacobi equation;
				%\\ \hspace*{4.25em} full-fledged Jacobi equation;
				\item[\smark] Jacobi equation \emph{up to homotopies}.
			\end{itemize}			
			}
		\end{column}	
	\end{columns}
\end{frame}
%-------------------------------------------------------------------------------------------------------------------------------------------------


%-----------------------------------------------------------%
\begin{frame}[fragile,shrink]{Details of the mapping cone construction \cite{Fiorenza2006}}
	\begin{displaymath}
		\begin{tikzcd}[ampersand replacement=\&,
			/tikz/execute at end picture={
    			\node  (large) [label=right:{\it dglas},draw,rectangle, draw,dashed, fit=(A1) (A2)] {};
    			\node (large2) [label=right:{\it dgvspaces},rectangle, draw,dashed, fit=(B1) (B2)] {};}] 
			|[alias=A1]| TW(f) \ar[dr] \ar[rr] \ar[drr,phantom,near start,"\lrcorner"]\& \& 0 \ar[d] \\
			\& X \arrow[r,""{name=U, below, draw=red}] \& |[alias=A2]| Y  \\
			\\
			\& X \ar[r,""{name=D, draw=red}] \arrow[from=U, to=D, rightsquigarrow]\& |[alias=B2]| Y  \\
			|[alias=B1]| MapCoCone(f) \ar[ur] \ar[rr] \ar[rru,phantom,near start,"\urcorner"]\& \& 0 \ar[u]
		\end{tikzcd}
	\end{displaymath}
	\vfill
	\begin{enumerate}
		\item $TW(f)$ Thom-Whitney of $X\to Y$, is a model for the homotopy fiber of $X\to Y$ in dglas \cite[Def 6.1.4]{Manetti2022b};
		\item forgetting the Lie structure, $MapCoCone(f)$ is a model for the homotopy fiber of $X\to Y$ in dgvspaces \cite[Def.5.1.3]{Manetti2022b};
		\item $TW(f)$ and $MapCoCone(f)$ are quasi isomorphic as dgspaces \cite[Cor. 6.1.7]{Manetti2022b}.
		\item Via homotopy transfer one gets an $L_\infty$-structure on $MapCoCone(f)$.
		\item the corresponding $L_\infty$-algebra is quasi-isomorphic to $TW(f)$ hence is a model for the homotopy fiber of the dgla map $X\to Y$.
	\end{enumerate}
\end{frame}
%-----------------------------------------------------------%

%-----------------------------------------------------------%
\ifstandalone
% https://en.wikibooks.org/wiki/LaTeX/Bibliographies_with_biblatex_and_biber
\begin{frame}[t,allowframebreaks]{Partial Bibliography}
	\nocite{Miti2021}
	\bibliographystyle{alpha}
	\bibliography{bibfile}
\end{frame}
\fi
%-----------------------------------------------------------%


%===========================================================%
\end{document}
%===========================================================%
