%+----------------------------------------------------------+
%| SLIDES: DF talk in Xian
%| Contents:	- 60 minutes (estimated duration: ~3 minutes per slide)
%| Author: Domenico Fiorenza
%| Comment: Transcription of a talk given by Domenico Fiorenza in China. 
%|          I am reproducing it faithfully to help myself move forward. I am a bit stuck!!!!
%+----------------------------------------------------------+


%- HandOut Flag --------------------------------------------+
\makeatletter
\@ifundefined{ifHandout}{%
  \expandafter\newif\csname ifHandout\endcsname
}{}
\makeatother

%- D0cum3nt ------------------------------------------------+
\documentclass[beamer,10pt]{standalone}   
%\documentclass[beamer,10pt,handout]{standalone}  \Handouttrue  

\ifHandout
	\setbeameroption{show notes} %print notes   
\fi

	
%- Packages ----------------------------------------------------------------------------------------------
\usepackage{custom-style}
\usepackage{math}




%--Beamer Style-----------------------------------------------------------------------------------------------
\usetheme{toninus}
\usepackage{animate}
\usetikzlibrary{positioning, arrows}
\usetikzlibrary{shapes}

%-- Custom Commands ---------------------------------------%
\newcommand\blfootnote[1]{%
  \begingroup
  \renewcommand\thefootnote{}\footnote{#1}%
  \addtocounter{footnote}{-1}%
  \endgroup
}
\newcommand{\seprule}{\par\noindent\rule{.4\textwidth}{0.4pt}\\
}

%===========================================================%
\begin{document}
%===========================================================%


%-----------------------------------------------------------%
\begin{frame}{1-plectic case}
	$M$ manifold; $\sigma \in \Omega^2_{\mathrm{cl}}(M)$ closed 2-form;
	\vfill

	Hamiltonian vector fields:
	$\X_{\sigma,\ham} \subseteq \X$
	$$ \X_{\sigma,\ham} = \{ X \in \X \mid \iota_X \sigma ~~\textrm{is exact }\}$$
	\blfootnote{i.e. $\exists f \in \Omega^0$ s.t. $\iota_X \sigma + \d f=0$}
	\vfill

	Hamiltonian pairs:
	$$\Ham^{(0)}_\sigma = \{(X,f) \mid \iota_X \sigma + \d f = 0 \}
	\subseteq \X\oplus \Omega^0 = \Gamma(E^{(0)})$$
	\vfill

	Standard Lie algebroid:
	$$ E^{(0)} = TM \oplus \R = TM \oplus \wedge^0 T^* M \to M$$

\end{frame}
%-----------------------------------------------------------%

%-----------------------------------------------------------%
\begin{frame}
	$\Ham^{(0)}_\sigma$ has a Lie algebra structure:
	\begin{equation}\label{eq:Bracket-HamPairs}
		\{ (X,f), (Y,g)\} = ( [X,Y], \sigma(X,Y))
	\end{equation}
	\footnote{$\sigma(X,Y)$ is an Hamiltonian 1-form for $[X,Y]$}
	\vfill

	Clearly, $\Ham^{(0)}_\sigma \twoheadrightarrow \X_{\sigma,\ham}$
	\\
	if $\sigma$ is non-degenerate, then $\Ham^{(0)}_\sigma \cong \Omega^0$ and \eqref{eq:Bracket-HamPairs} is the usual \emph{Poisson bracket}.
	\footnote{Note: $\sigma(X,Y) = \iota_Y \iota_X \sigma = -\iota_Y \d f = \iota_X \d g$ for $(X,f), (Y,g) \in \Ham^{(0)}_\sigma$.}
	\vfill

	\seprule
	$\X\oplus \Omega^0$ carries a Lie algebra structure:
	\begin{equation}\label{eq:Bracket-LieAlgoid}
		\{ (X,f), (Y,g)\} = ( [X,Y], \Lie_X g - \Lie_Y f + \sigma(X,Y) )
	\end{equation}

\end{frame}
%-----------------------------------------------------------%

%-----------------------------------------------------------%
\begin{frame}
	\begin{block}{Notations:}
		\begin{itemize}
			\item $\Courant[\sigma]{0} := \X\oplus \Omega^{0}$ with its Lie algebra structure \eqref{eq:Bracket-LieAlgoid}
			\item $\Rogers[\sigma]{0} := \Ham^{(0)}_\sigma$ with its Lie algebra structure \eqref{eq:Bracket-HamPairs}
		\end{itemize}
	\end{block}
	\vfill

	\begin{displaymath}
		\begin{tikzcd}[ampersand replacement=\&]
			\Rogers[\sigma]{0}   \ar[r,two heads]\ar[d,hook]\& \X_{\sigma,\ham}
			\\ 
			\Courant[\sigma]{0}
		\end{tikzcd}
	\end{displaymath}
	\vfill

	\seprule
	this was the $r=1$ case : $\sigma$ is a degree $r+1$ closed form and $\Rogers[\sigma]{r},\Courant[\sigma]{r}$ are $L_r$-algebras.
	\\
	Let us now move from $r=1$ to $r=2$.

\end{frame}
%-----------------------------------------------------------%

%-----------------------------------------------------------%
\begin{frame}{2-plectic case}
	$\sigma \in \Omega^3_{\mathrm{cl}}(M)$ closed 3-form;
	\vfill

	Hamiltonian vector fields: $\X_{\sigma,\ham}$ same definition as before
	\vfill

	Hamiltonian pairs:
	$$
		\Ham^{(1)}_\sigma = \{ (X,\omega) \mid \iota_X \sigma + \d \omega  = 0 \}\subseteq \X\oplus \Omega^1 = \Gamma(E^{(1)})
	$$
	\vfill
	
	Standard Courant algebroid:
	$$ E^{(1)} = TM \oplus T^*M = TM \oplus \wedge^1 T^* M \to M$$
	\vfill

	\emph{do $\Ham^{(1)}_\sigma$ and $E^{(1)}$ carry a Lie algebra structure?}
	\vfill

	$\Gamma(E^{(1)})$ has a canonical antisymmetric bracket (called \emph{Courant bracket}), that however does \underline{not} satisfy the Jacobi identity.
\end{frame}
%-----------------------------------------------------------%

%-----------------------------------------------------------%
\begin{frame}
	Yet it satisfies it \emph{"up to homotopy"}. This suggests there could be a $L_\infty$-algebra structure hiding here.
	\vfill

	\seprule
	\begin{displaymath}
		\begin{tikzcd}[ampersand replacement=\&]
			\& \text{\small (deg $-1$)} \& \text{\small (deg $0$)} \\[-.5em]
			\Rogers[\sigma]{1}   \ar[r,phantom,":="]
			\&
			\Omega^0 \ar[r,"{(0,\d)}"] \ar[d,"id"]
			\& \Ham_{\sigma}^{(1)} \ar[d,hook]
			\&[1em] \parbox{10em}{carries a distinguished $L_\infty$-algebra structure}
			\\
			\Courant[\sigma]{1} \ar[r,phantom,":="]
			\& \Omega^0 \ar[r,"{(0,\d)}"]
			\& \Gamma(E^{(1)}) \&
			\parbox{10em}{forms a $L_2$-algebra with the courant bracket [Roytenberg-Weinstein]}
		\end{tikzcd}
	\end{displaymath}
	\vfill

	Vertical arrows are the linear part of a $L_\infty$-morphism $\Rogers[\sigma]{1} \to \Courant[\sigma]{1}$.
\end{frame}
%-----------------------------------------------------------%

%-----------------------------------------------------------%
\begin{frame}{General picture: arbitrary r}
	$\sigma \in \Omega^{r+1}_{\mathrm{cl}}(M)$, 
	$\X_{\sigma,\ham}$ same definition.
	\vfill

	Hamiltonian pairs:
	$$
		\Ham^{(r-1)}_\sigma = 
		\{ (X,\omega) \mid \iota_X \sigma + \d \omega  = 0 \}
		\subseteq \X\oplus \Omega^{r-1} = \Gamma(E^{(r-1)})
	$$
	\vfill
	
	Standard higher Courant (Vinogradov) algebroid:
	$$ E^{(r-1)} = TM \oplus \wedge^{r-1} T^* M \to M$$
	\vfill

	\begin{displaymath}
		\begin{tikzcd}[ampersand replacement=\&,column sep = small]
			\& \text{\small (deg $1-r$)} \& \cdots \& \&\text{\small (deg $-1$)} \& \text{\small (deg $0$)} \\[-.5em]
			\Rogers[\sigma]{r-1}   \ar[r,phantom,":="]
			\&
			\Omega^0 \ar[r,"{\d}"] \ar[d,"id"]
			\&
			\Omega^1 \ar[r,"{\d}"] \ar[d,"id"]
			\&
			\cdots
			\&
			\Omega^{r-2} \ar[r,"{(0,\d)}"] \ar[d,"id"]
			\& \Ham_{\sigma}^{(r-1)} \ar[d,hook]
			\\
			\Courant[\sigma]{r-1} \ar[r,phantom,":="]
			\&
			\Omega^0 \ar[r,"{\d}"]
			\&
			\Omega^1 \ar[r,"{\d}"] 
			\&
			\cdots
			\&
			\Omega^{r-2} \ar[r,"{(0,\d)}"]
			\& \Gamma(E^{(1)})
		\end{tikzcd}
	\end{displaymath}

\end{frame}
%-----------------------------------------------------------%

%-----------------------------------------------------------%
\begin{frame}

\end{frame}
%-----------------------------------------------------------%


%===========================================================%
\end{document}
%===========================================================%