%+----------------------------------------------------------+
%| SLIDES: DF talk in Xian
%| Contents:	- 60 minutes (estimated duration: ~3 minutes per slide)
%| Author: Domenico Fiorenza
%| Comment: Transcription of a talk given by Domenico Fiorenza in China. 
%|          I am reproducing it faithfully to help myself move forward. I am a bit stuck!!!!
%+----------------------------------------------------------+


%- HandOut Flag --------------------------------------------+
\makeatletter
\@ifundefined{ifHandout}{%
  \expandafter\newif\csname ifHandout\endcsname
}{}
\makeatother

%- D0cum3nt ------------------------------------------------+
\documentclass[beamer,10pt]{standalone}   
%\documentclass[beamer,10pt,handout]{standalone}  \Handouttrue  

\ifHandout
	\setbeameroption{show notes} %print notes   
\fi

	
%- Packages ----------------------------------------------------------------------------------------------
\usepackage{custom-style}
\usepackage{math}




%--Beamer Style-----------------------------------------------------------------------------------------------
\usetheme{toninus}
\usepackage{animate}
\usetikzlibrary{positioning, arrows}
\usetikzlibrary{shapes}

%-- Custom Commands ---------------------------------------%


\usetikzlibrary{fit}

%===========================================================%
\begin{document}
%===========================================================%


%-----------------------------------------------------------%
\begin{frame}{1-plectic case}
	$M$ manifold; $\sigma \in \Omega^2_{\mathrm{cl}}(M)$ closed 2-form;
	\vfill

	Hamiltonian vector fields:
	$\X_{\sigma,\ham} \subseteq \X$
	$$ \X_{\sigma,\ham} = \{ X \in \X \mid \iota_X \sigma ~~\textrm{is exact }\}$$
	\blfootnote{i.e. $\exists f \in \Omega^0$ s.t. $\iota_X \sigma + \d f=0$}
	\vfill

	Hamiltonian pairs:
	$$\Ham^{(0)}_\sigma = \{(X,f) \mid \iota_X \sigma + \d f = 0 \}
	\subseteq \X\oplus \Omega^0 = \Gamma(E^{(0)})$$
	\vfill

	Standard Lie algebroid:
	$$ E^{(0)} = TM \oplus \R = TM \oplus \wedge^0 T^* M \to M$$

\end{frame}
%-----------------------------------------------------------%

%-----------------------------------------------------------%
\begin{frame}
	$\Ham^{(0)}_\sigma$ has a Lie algebra structure:
	\begin{equation}\label{eq:Bracket-HamPairs}
		\{ (X,f), (Y,g)\} = ( [X,Y], \sigma(X,Y))
	\end{equation}
	\footnote{$\sigma(X,Y)$ is an Hamiltonian 1-form for $[X,Y]$}
	\vfill

	Clearly, $\Ham^{(0)}_\sigma \twoheadrightarrow \X_{\sigma,\ham}$
	\\
	if $\sigma$ is non-degenerate, then $\Ham^{(0)}_\sigma \cong \Omega^0$ and \eqref{eq:Bracket-HamPairs} is the usual \emph{Poisson bracket}.
	\footnote{Note: $\sigma(X,Y) = \iota_Y \iota_X \sigma = -\iota_Y \d f = \iota_X \d g$ for $(X,f), (Y,g) \in \Ham^{(0)}_\sigma$.}
	\vfill

	\seprule
	$\X\oplus \Omega^0$ carries a Lie algebra structure:
	\begin{equation}\label{eq:Bracket-LieAlgoid}
		\{ (X,f), (Y,g)\} = ( [X,Y], \Lie_X g - \Lie_Y f + \sigma(X,Y) )
	\end{equation}

\end{frame}
%-----------------------------------------------------------%

%-----------------------------------------------------------%
\begin{frame}
	\begin{block}{Notations:}
		\begin{itemize}
			\item $\Courant[\sigma]{0} := \X\oplus \Omega^{0}$ with its Lie algebra structure \eqref{eq:Bracket-LieAlgoid}
			\item $\Rogers[\sigma]{0} := \Ham^{(0)}_\sigma$ with its Lie algebra structure \eqref{eq:Bracket-HamPairs}
		\end{itemize}
	\end{block}
	\vfill

	\begin{displaymath}
		\begin{tikzcd}[ampersand replacement=\&]
			\Rogers[\sigma]{0}   \ar[r,two heads]\ar[d,hook]\& \X_{\sigma,\ham}
			\\ 
			\Courant[\sigma]{0}
		\end{tikzcd}
	\end{displaymath}
	\vfill

	\seprule
	this was the $r=1$ case : $\sigma$ is a degree $r+1$ closed form and $\Rogers[\sigma]{r},\Courant[\sigma]{r}$ are $L_r$-algebras.
	\\
	Let us now move from $r=1$ to $r=2$.

\end{frame}
%-----------------------------------------------------------%

%-----------------------------------------------------------%
\begin{frame}{2-plectic case}
	$\sigma \in \Omega^3_{\mathrm{cl}}(M)$ closed 3-form;
	\vfill

	Hamiltonian vector fields: $\X_{\sigma,\ham}$ same definition as before
	\vfill

	Hamiltonian pairs:
	$$
		\Ham^{(1)}_\sigma = \{ (X,\omega) \mid \iota_X \sigma + \d \omega  = 0 \}\subseteq \X\oplus \Omega^1 = \Gamma(E^{(1)})
	$$
	\vfill
	
	Standard Courant algebroid:
	$$ E^{(1)} = TM \oplus T^*M = TM \oplus \wedge^1 T^* M \to M$$
	\vfill

	\emph{do $\Ham^{(1)}_\sigma$ and $E^{(1)}$ carry a Lie algebra structure?}
	\vfill

	$\Gamma(E^{(1)})$ has a canonical antisymmetric bracket (called \emph{Courant bracket}), that however does \underline{not} satisfy the Jacobi identity.
\end{frame}
%-----------------------------------------------------------%

%-----------------------------------------------------------%
\begin{frame}
	Yet it satisfies it \emph{"up to homotopy"}. This suggests there could be a $L_\infty$-algebra structure hiding here.
	\vfill

	\seprule
	\begin{displaymath}
		\begin{tikzcd}[ampersand replacement=\&]
			\& \text{\small (deg $-1$)} \& \text{\small (deg $0$)} \\[-.5em]
			\Rogers[\sigma]{1}   \ar[r,phantom,":="]
			\&
			\Omega^0 \ar[r,"{(0,\d)}"] \ar[d,"id"]
			\& \Ham_{\sigma}^{(1)} \ar[d,hook]
			\&[1em] \parbox{10em}{carries a distinguished $L_\infty$-algebra structure}
			\\
			\Courant[\sigma]{1} \ar[r,phantom,":="]
			\& \Omega^0 \ar[r,"{(0,\d)}"]
			\& \Gamma(E^{(1)}) \&
			\parbox{10em}{forms a $L_2$-algebra with the courant bracket \cite{Roytenberg1998}}
		\end{tikzcd}
	\end{displaymath}
	\vfill

	Vertical arrows are the linear part of a $L_\infty$-morphism $\Rogers[\sigma]{1} \to \Courant[\sigma]{1}$.
\end{frame}
%-----------------------------------------------------------%

%-----------------------------------------------------------%
\begin{frame}{General picture: arbitrary r}
	$\sigma \in \Omega^{r+1}_{\mathrm{cl}}(M)$, 
	$\X_{\sigma,\ham}$ same definition.
	\vfill

	Hamiltonian pairs:
	$$
		\Ham^{(r-1)}_\sigma = 
		\{ (X,\omega) \mid \iota_X \sigma + \d \omega  = 0 \}
		\subseteq \X\oplus \Omega^{r-1} = \Gamma(E^{(r-1)})
	$$
	\vfill
	
	Standard higher Courant (Vinogradov) algebroid:
	$$ E^{(r-1)} = TM \oplus \wedge^{r-1} T^* M \to M$$
	\vfill

	\begin{displaymath}
		\begin{tikzcd}[ampersand replacement=\&,column sep = small]
			\& \text{\small (deg $1-r$)} \& \cdots \& \&\text{\small (deg $-1$)} \& \text{\small (deg $0$)} \\[-.5em]
			\Rogers[\sigma]{r-1}   \ar[r,phantom,":="]
			\&
			\Omega^0 \ar[r,"{\d}"] \ar[d,"id"]
			\&
			\Omega^1 \ar[r,"{\d}"] \ar[d,"id"]
			\&
			\cdots
			\&
			\Omega^{r-2} \ar[r,"{(0,\d)}"] \ar[d,"id"]
			\& \Ham_{\sigma}^{(r-1)} \ar[d,hook]
			\\
			\Courant[\sigma]{r-1} \ar[r,phantom,":="]
			\&
			\Omega^0 \ar[r,"{\d}"]
			\&
			\Omega^1 \ar[r,"{\d}"] 
			\&
			\cdots
			\&
			\Omega^{r-2} \ar[r,"{(0,\d)}"]
			\& \Gamma(E^{(1)})
		\end{tikzcd}
	\end{displaymath}
	\vfill

	\begin{itemize}
		\item the vertical arrows give a morphism of cochain complexes.
		\item \cite{Rogers2010} endows $\Rogers[\sigma]{r-1}$ with a $L_\infty$-algebra structure.
		\item \cite{Zambon2012} endows $\Courant[\sigma]{r-1}$ with a $L_\infty$-algebra structure.
	\end{itemize}
	\vfill
\end{frame}
%-----------------------------------------------------------%

%-----------------------------------------------------------%
\begin{frame}{Zambon's construction}
	Consider the graded manifold $T^*[r]T[1]M$.
	\vfill

	$C^\infty(T^*[r]T[1]M)$ is a \emph{$r$-Poisson algebra} (associative multiplication, degree $r$ Poisson bracket $\lbrace \cdot,\cdot\rbrace$)\footnote{Canonical Poisson bracket on the shifted cotangent bundle}
	\vfill

	$\mathcal{C}_r := C^\infty(T^*[r]T[1]M)[r]$ is a graded Lie algebra.
	\vfill

	$\mathcal{C}_r$ contains all differntial forms on $M$ since:
	\begin{displaymath}
		\begin{tikzcd}[ampersand replacement=\&,column sep = large]
			T^*[r]T[1]M \ar[r, two heads,] \& T[1]M \&[-2.5em] \\[-.5em]
			C^\infty(T^*[r]T[1]M) \& C^\infty(T[1]M) \ar[l,hook]\ar[r,phantom,"\cong"] \& \Omega(M) 
 		\end{tikzcd}
	\end{displaymath}
\end{frame}
%-----------------------------------------------------------%

%-----------------------------------------------------------%
\begin{frame}
	There exists a distinguished element $\mathcal{S}$ of degree $r+1$ in $C^\infty(T^*[r]T[1]M)$ such that $\mathcal{C}_r$ such that $\lbrace \mathcal{S},\mathcal{S}\rbrace =0 $, lifting the \emph{de Rham} differential of $M$ to $C^\infty(T^*[r]T[1]M)$ given by 
	$$ \delta := \lbrace \mathcal{S},\cdot \rbrace$$
	\vfill

	$\mathcal{S}$ has degree $1$ in $\mathcal{C}_r$ (MC element) therefore $(\mathcal{C}_r,\lbrace \cdot,\cdot\rbrace,\delta)$ is a dg Lie algebra.
	\vfill

	Consider the embedding: $$ \mathcal{C}_r^{\geq 0} \hookrightarrow \mathcal{C}_r$$
	\begin{enumerate}
		\item \cite{Getzler1991} construct a canonical $L_\infty$-algebra structure on $\mathcal{C}_r^{<0}\cong \frac{\mathcal{C}_r }{\mathcal{C}_r^{\geq 0}}$ (via \emph{derived brackets});
		\item \cite{Fiorenza2006} construct a canonical $L_\infty$-algebra associated with any dg Lie algebra morphism $\g \to \mathfrak{h}$ (via \emph{mapping cones});
		\item \cite{Pridham2010a} showed that the $L_\infty$-algebra  of 2. is a \emph{model for the homotopy fiber $\hofib(\g \to \mathfrak{h})$}.
		\item \cite{Bandiera2015} showed that 1. is a model for the homotopy fiber $\hofib(\mathcal{C}_r^{\geq 0} \hookrightarrow \mathcal{C}_r)$.
	\end{enumerate}
\end{frame}
%-----------------------------------------------------------%

%-----------------------------------------------------------%
\begin{frame}[fragile,shrink]{Details of the mapping cone construction}
	\begin{displaymath}
		\begin{tikzcd}[ampersand replacement=\&,
			/tikz/execute at end picture={
    			\node  (large) [label=right:{\it dglas},draw,rectangle, draw,dashed, fit=(A1) (A2)] {};
    			\node (large2) [label=right:{\it dgvspaces},rectangle, draw,dashed, fit=(B1) (B2)] {};}] 
			|[alias=A1]| TW(f) \ar[dr] \ar[rr] \ar[drr,phantom,near start,"\lrcorner"]\& \& 0 \ar[d] \\
			\& X \arrow[r,""{name=U, below, draw=red}] \& |[alias=A2]| Y  \\
			\\
			\& X \ar[r,""{name=D, draw=red}] \arrow[from=U, to=D, rightsquigarrow]\& |[alias=B2]| Y  \\
			|[alias=B1]| MapCoCone(f) \ar[ur] \ar[rr] \ar[rru,phantom,near start,"\urcorner"]\& \& 0 \ar[u]
		\end{tikzcd}
	\end{displaymath}
	\vfill
	\begin{enumerate}
		\item $TW(f)$ Thom-Whitney of $X\to Y$, is a model for the homotopy fiber of $X\to Y$ in dglas \cite[Def 6.1.4]{Manetti2022b};
		\item forgetting the Lie structure, $MapCoCone(f)$ is a model for the homotopy fiber of $X\to Y$ in dgvspaces \cite[Def.5.1.3]{Manetti2022b};
		\item $TW(f)$ and $MapCoCone(f)$ are quasi isomorphic as dgspaces \cite[Cor. 6.1.7]{Manetti2022b}.
		\item Via homotopy transfer one gets an $L_\infty$-structure on $MapCoCone(f)$.
		\item the corresponding $L_\infty$-algebra is quasi-isomorphic to $TW(f)$ hence is a model for the homotopy fiber of the dgla map $X\to Y$.
	\end{enumerate}
\end{frame}
%-----------------------------------------------------------%

%-----------------------------------------------------------%
\begin{frame}{What to read out of Pridham2010a}
	nell'articolo di Pridham i risultati di interesse sono il Lemma 3.24, la Proposizione 4.42 e il Corollario 4.57 (numerazione dell'ultima versione arXiv)

\end{frame}
%-----------------------------------------------------------%



%-----------------------------------------------------------%
\begin{frame}
	Consider $\mathcal{C}_r^{\geq 0}[-1] \hookrightarrow \mathcal{C}_r[-1]$
	\vfill

	Observe that $\mathcal{C}_r^{<0} = C^\infty(T^*[r]T[1]M)^{<r}=$
	\begin{displaymath}
		\begin{tikzcd}[ampersand replacement=\&]
			\Omega^0 \ar[r,"\d"]\&
			\Omega^1 \ar[r,"\d"]\&
			\Omega^2 \ar[r,"\d"]\&
			\cdots \ar[r,"\d"]\&
			\Omega^{r-2} \ar[r,"{(\d,0)}"]\&
			\Omega^{r-1}\ar[d,phantom,"\oplus"]\\[-.7em]
			\&\&\&\&\&\X
 		\end{tikzcd}
	\end{displaymath}
	\vfill

	\begin{displaymath}
		\begin{tikzcd}[ampersand replacement=\&,column sep = small]
			\& \text{\small (deg $1-r$)} \& \cdots \& \&\text{\small (deg $-1$)} \& \text{\small (deg $0$)} \\[-.5em]
			\Rogers[\sigma]{r-1}   \ar[r,phantom,":="]
			\&
			\Omega^0 \ar[r,"{\d}"] \ar[d,"id"]
			\&
			\Omega^1 \ar[r,"{\d}"] \ar[d,"id"]
			\&
			\cdots
			\&
			\Omega^{r-2} \ar[r,"{(0,\d)}"] \ar[d,"id"]
			\& \Ham_{\sigma}^{(r-1)} \ar[d,hook]
			\\
			\Courant[\sigma]{r-1} \ar[r,phantom,":="]
			\&
			\Omega^0 \ar[r,"{\d}"]
			\&
			\Omega^1 \ar[r,"{\d}"] 
			\&
			\cdots
			\&
			\Omega^{r-2} \ar[r,"{(0,\d)}"]
			\& \Gamma(E^{(1)})
		\end{tikzcd}
	\end{displaymath}
	\vfill

	\cite{Miti2021} proved that the vertical arrows are the linear part of a $L_\infty$-morphism $\Rogers[\sigma]{r-1} \to \Courant[\sigma]{r-1}$.
\end{frame}
%-----------------------------------------------------------%

%-----------------------------------------------------------%
\begin{frame}{a more exotic morphism}
	For $r=1,2$ we have:
	\begin{displaymath}
		\begin{tikzcd}[ampersand replacement=\&,column sep = small]
			\Courant[\sigma]{0} \ar[r,phantom,":="] \ar[d]
			\&[1em]
			0 \ar[r] \ar[d]
			\& \Omega^0\oplus \X \ar[d,"{(\d,\id)}"]
			\\
			\Courant{1} \ar[r,phantom,":="] 
			\&
			\Omega^0 \ar[r,"{(\d,0)}"] 
			\& 
			\Omega^1\oplus \X
		\end{tikzcd}
	\end{displaymath}
	\cite{Zambon2012} proved that  the vertical arrows are the linear part of a $L_\infty$-morphism $\Courant[\sigma]{0} \to \Courant[\sigma]{1}$.
	\vfill

	More generally we have a morphism of chain complexes:
	\begin{displaymath}
		\begin{tikzcd}[ampersand replacement=\&,column sep = small]
			\Courant[\sigma]{r-1} %\ar[r,phantom,":="] \ar[d]
			\&[1em]
			0 \ar[r] \ar[d]
			\& \Omega^0 \ar[d,"\d"] \ar[r]
			\& \Omega^1 \ar[d,"\d"] \ar[r]
			\& \cdots \ar[r]
			\& \Omega^{r-2} \ar[d,"\d"] \ar[r,"{(\d,0)}"]
			\&[.5em]  \Omega^{r-1} \oplus \X \ar[d,"{(\d,\id)}"]
			\\
			\Courant{r} %\ar[r,phantom,":="]
			\& \Omega^0 \ar[r]
			\& \Omega^1 \ar[r]
			\& \Omega^2 \ar[r]
			\& \cdots \ar[r]
			\& \Omega^{r-2} \ar[r,"{(\d,0)}"]
			\&\Omega^{r-1} \oplus \X
		\end{tikzcd}
	\end{displaymath}
	Is this the linear part of a $L_\infty$-morphism $\Courant[\sigma]{r-1} \to \Courant{r}$?
	\vfill
\end{frame}
%-----------------------------------------------------------%

%-----------------------------------------------------------%
\begin{frame}
	Yes! (Fiorenza-Miti 2025)
	\vfill

	\begin{block}{Idea:}
		We want to exhibit a morphism:
		\begin{displaymath}
			\begin{tikzcd}[ampersand replacement=\&]
				\hofib(\mathcal{C}_{r,\sigma}^{\geq 0} \hookrightarrow \mathcal{C}_{r,\sigma}) \ar[d]
				\\
				\hofib(\mathcal{C}_{r+1}^{\geq 0} \hookrightarrow \mathcal{C}_{r+1})
			\end{tikzcd}
		\end{displaymath}
		This should be naturally induced by a (homotopy) commutative diagram of dg-Lie algebars (or $L\infty$-algebras):
		\begin{displaymath}
			\begin{tikzcd}[ampersand replacement=\&]
				\mathcal{C}_{r,\sigma}^{\geq 0} \ar[r,hook] \ar[d ]
				\& \mathcal{C}_{r,\sigma} \ar[d, ]
				\\
				\mathcal{C}_{r+1}^{\geq 0} \ar[r,hook] \& \mathcal{C}_{r+1} 
			\end{tikzcd}
		\end{displaymath}
	\end{block}
\end{frame}
%-----------------------------------------------------------%

%-----------------------------------------------------------%
\begin{frame}
  Inside $\mathcal{C}_r$ there is a smaller, mora manageble, dg-Lie subalgebra with the same negative part.
  \vfill

  Consider the $2$-terms dg-Lie algebra $\Cartan \subseteq \Der(\Omega(M))$ given by:
  \begin{displaymath}
	\begin{tikzcd}[ampersand replacement=\&,row sep = small]
		\text{\tiny (deg $-1$)} \& \text{\tiny (deg $0$)}\&[2em]
		\\[-.7em]
		\iota_{\X} \ar[r,"\delta"] \& \Lie_{\X} \ar[r,phantom,";"] \& \delta:= [\d, \cdot]
	\end{tikzcd}
  \end{displaymath}
  \vfill

  $\Omega$ is a module for $\Der(\Omega)$ and so it is a module for $\Cartan$.
  \\
  By shifting degrees, $\Omega[r]$ is a module for $\Cartan$.
\end{frame}
%-----------------------------------------------------------%

%-----------------------------------------------------------%
\begin{frame}
	We can form the dg-Lie algebra
	$$ \g_r := \Cartan \ltimes \Omega[r]$$
	\vfill

	Observe that any $\sigma \in \Omega^{r+1}_{\mathrm{cl}}(M)$ is a MC (degree 1) element of $\g_r$.
	\vfill
	
	We can consider the twisted dg-Lie algebra $(\g_{r,\sigma}, \d_\sigma, \lbrace \cdot,\cdot\rbrace_\sigma)$ with:
	\begin{displaymath}
		\d_\sigma (\omega) = \d \omega~, \qquad
		\d_\sigma (\iota_X) = \Lie_X \iota_X \sigma~, \qquad
		\d_\sigma(\Lie_X) = -\Lie_X\sigma
	\end{displaymath}
	\vfill

	We have
	\begin{displaymath}
		\begin{tikzcd}[ampersand replacement=\&]
			\g_{r,\sigma}^{~\geq 0} \ar[r,hook] \ar[d,hook] \&
			\g_{r,\sigma} \ar[d,hook] \ar[r,two heads] \&[1em]
			\g_{r,\sigma}^{~< 0} \ar[d,equal] \\
			\mathcal{C}_{r,\sigma}^{~\geq 0} \ar[r,hook] \&
			\mathcal{C}_{r,\sigma} \ar[r,two heads] \&
			\mathcal{C}_{r,\sigma}^{~< 0}
		\end{tikzcd}
	\end{displaymath}
	\vfill

	Therefore $\mathcal{C}_{r,\sigma}^{~< 0}[-1] = \g_{r,\sigma}^{~<0}[-1]$ as $L_\infty$-algebras with the derived brackets.
\end{frame}
%-----------------------------------------------------------%

%-----------------------------------------------------------%
\begin{frame}
  So we can replace $\mathcal{C}_{r,\sigma}, \mathcal{C}_{r+1}$ with $\g_{r,\sigma}, \g_{r+1}$ in the previous commuative square:
	\begin{displaymath}
		\begin{tikzcd}[ampersand replacement=\&]
			\g_{r,\sigma}^{~\geq 0} \ar[r,hook] \ar[d, ]
			\& \g_{r,\sigma} \ar[d, ]
			\\
			\g_{r+1}^{~\geq 0} \ar[r,hook] \& \g_{r+1}
		\end{tikzcd}
	\end{displaymath}
	\vfill

	Moreover, if this has to be true for any $\sigma \in \Omega^{r+1}_{\mathrm{cl}}(M)$, it has to be true for $\sigma = 0$. So let begin with the simpler situation
	\begin{displaymath}
		\begin{tikzcd}[ampersand replacement=\&]
			\g_{r}^{~\geq 0} \ar[r,hook] \ar[d, ]
			\& \g_{r} \ar[d, ]
			\\
			\g_{r+1}^{~\geq 0} \ar[r,hook] \& \g_{r+1}
		\end{tikzcd}
	\end{displaymath}
	\vfill
	Let us consider the right-hand vertical arrow:
\end{frame}
%-----------------------------------------------------------%

%-----------------------------------------------------------%
\begin{frame}
		The vertical arrows forms a morphism of cochain complexes $\Phi_1:\g_{r,\sigma}\to \g_{r+1}$:
	\begin{displaymath}
		\begin{tikzcd}[ampersand replacement=\&, column sep = small]
			0\ar[r]\ar[d] \&
			\Omega^0 \ar[r] \ar[d,"\d"] \&
			\Omega^1 \ar[r] \ar[d,"\d"] \&
			\Omega^2 \ar[r] \&
			\cdots \ar[r] \&
			\Omega^{r-1}\oplus \iota_{\X} \ar[d,"\d\oplus \id"] \ar[r] \&
			\Omega^{r}\oplus \Lie_{\X} \ar[r]\ar[d,"\d\oplus\id"] \&
			\Omega^{r+1} \ar[d,"\d"] \ar[r,phantom,"\cdots"] \& \phantom{.}
			\\
			\Omega^{0} \ar[r] \&
			\Omega^{1} \ar[r] \&
			\Omega^{2} \ar[r] \&
			\cdots \& \cdots \ar[r] \&
			\Omega^{r}\oplus\iota_{\X} \ar[r] \&
			\Omega^{r+1}\oplus \Lie_{\X} \ar[r] \&
			\Omega^{r+2} \ar[r,phantom,"\cdots"]  \& \phantom{.}
		\end{tikzcd}
	\end{displaymath}
	%
	\vfill

	Is $\Phi_1$ a dg-Lie algebra morphism?
	$$ \Phi_1\big(\lbrace a, b \rbrace\big) \overset{?}{=} \big\lbrace \Phi_1(a), \Phi_1(b) \big\rbrace$$
	\vfill

	No: the de Rham differential does not commute with the contractions:
	$$ \Phi_1\big(\lbrace \iota_X, \omega \rbrace\big) \neq \big\lbrace \Phi_1(\iota_X), \Phi_1(\omega)\big\rbrace~.$$
\end{frame}
%-----------------------------------------------------------%

%-----------------------------------------------------------%
\begin{frame}
	Let us try and cure this by setting
	\begin{align*}
		\Phi_2(\iota_X,\omega) &=~ \iota_X \omega \\
		\Phi_2(\omega,\iota_X) &=~ \pm \iota_X \omega \\
		\Phi_2(a,b) &=~ 0 \qquad \text{in all other cases}
	\end{align*}
	\vfill

	$\Phi= (\Phi_1,\Phi_2,0,\dots)$ is a $L_\infty$-morphism $\g_r \to \g_{r+1}$!
	\vfill

	Does it maps $\g_{r,\sigma}^{~\geq 0}$ to $\g_{r+1}^{~\geq 0}$? Yes!\footnote{ We need to check that $\Phi_2(a,b)\in \g_{r+1}^{~\geq 0}$ when $|a|=|b|=0$. this is true since for these elements $\Phi_2(a,b)=0$.}
	\vfill

	So we have a strictly commutative diagram of $L_\infty$-morphisms:
	\begin{displaymath}
		\begin{tikzcd}[ampersand replacement=\&]
			\g_{r,\sigma}^{~\geq 0} \ar[r,hook] \ar[d, "\Phi \vert_{\geq 0}"']
			\& \g_{r,\sigma} \ar[d, "\Phi"]
			\\
			\g_{r+1}^{~\geq 0} \ar[r,hook] \& \g_{r+1}
		\end{tikzcd}
	\end{displaymath}
\end{frame}
%-----------------------------------------------------------%

%-----------------------------------------------------------%
\begin{frame}
  What for arbitrary $\sigma \in \Omega^{r+1}_{\mathrm{cl}}(M)$?
  We use $\sigma$ to twist everything:
	\begin{displaymath}
		\begin{tikzcd}[ampersand replacement=\&]
			\g_{r,\sigma}^{~\geq 0} \ar[r,hook] \ar[d, "\Phi_\sigma \vert_{\geq 0}"']
			\& \g_{r,\sigma} \ar[d, "\Phi_\sigma"]
			\\
			\g_{r+1,\Phi(\sigma)}^{~\geq 0} \ar[r,hook] \& \g_{r+1,\Phi(\sigma)}
		\end{tikzcd}
	\end{displaymath}
	\vfill
	Where:
	\begin{align*}
		\big(\Phi_\sigma \big)_j (\cdots) 
		&=~
		\sum_{k=0}^\infty \frac{\Phi_{k+j}}{k!} (\overbrace{\sigma,\cdots,\sigma}^{\text{\tiny $k$ copies}},\cdots)
		\\
		\Phi(\sigma) &=~ \sum_{k=0}^\infty \frac{\Phi_k(\sigma,\cdots,\sigma)}{k!} = \d(\sigma) + \frac{1}{2}\cdot 0 = 0
	\end{align*}
\end{frame}
%-----------------------------------------------------------%

%-----------------------------------------------------------%
\begin{frame}
	So
	\begin{displaymath}
		\begin{tikzcd}[ampersand replacement=\&]
			\g_{r,\sigma}^{~\geq 0} \ar[r,hook] \ar[d, "\Phi_\sigma \vert_{\geq 0}"']
			\& \g_{r,\sigma} \ar[d, "\Phi_\sigma"]
			\\
			\g_{r+1 }^{~\geq 0} \ar[r,hook] \& \g_{r+1 }
		\end{tikzcd}
	\end{displaymath}
	\vfill
	\begin{block}{Details}
		\begin{itemize}
			\item we have to check that $\Phi_\sigma$ preseves the positive part. This is immediate since $(\Phi_\sigma)_2 = \Phi_2$.
			\item Moreover, we have 
			$$ (\Phi_\sigma)_1 \big\vert_{\geq 0} = \Phi_1 \big\vert_{\geq 0} + \frac{1}{2}\cancel{\Phi_2(\sigma,\cdot)\big\vert_{\geq 0}} = \Phi_1\big\vert_{\geq 0} $$
		\end{itemize}
	\end{block}
	\vfill
	
\end{frame}
%-----------------------------------------------------------%

%-----------------------------------------------------------%
\begin{frame}{The \cite{Miti2021} $L_\infty$-morphism revisited}
	We have the following:
	\begin{displaymath}
		\begin{tikzcd}[ampersand replacement=\&]
			\Rogers[\sigma]{r-1} \ar[d] \ar[r] \&
			\X_{\ham,\sigma} \ar[r,"\iota_{\dots}\sigma"] \ar[d,"\Lie"]\&
			(\Omega^0\to\Omega^1\to\cdots\to\Omega^{r-1}\to \d \Omega^{r-1}) \ar[d,hook] 
			\\
			\Courant[\sigma]{r-1} \ar[r] \&
			\g_{r+1,\sigma}^{~\geq 0} \ar[r,hook] \ar[ur,Rightarrow]\& 
			\g_{r+1,\sigma}
		\end{tikzcd}
	\end{displaymath}
	\vfill

	Since \cite{Fiorenza2014a} shows that:
	\begin{itemize}
		\item[$\bullet$] $\iota_{\ldots}\sigma$ (multicontractions with $\sigma$) is a $L_\infty$-morphism
		\item[$\bullet$] $\Rogers[\sigma]{r-1} =\hofib(\iota_{\ldots}\sigma)$\footnote{Is it better to say that is a model of the homotopy fiber?}
	\end{itemize}
	\vfill

	The homotopy is given by $$ e^{\iota} * \Lie = \iota_{\ldots}\sigma$$
	which is basically Cartan's magic formula for multivector fields\footnote{Twisting the differential, i.e. changing $\d$ in $\d_\sigma$, the multicontractions $\iota_{x_1\wedge\cdots\wedge x_k}\sigma$ appear.}
	$$[\d, \iota_{x_1\wedge\cdots\wedge x_k}] = \Lie_{x_1\wedge\cdots\wedge x_k}$$

\end{frame}
%-----------------------------------------------------------%



%-----------------------------------------------------------%
\begin{frame}{Conclusion:}

	\vfill

	\begin{displaymath}
		\begin{tikzcd}[ampersand replacement=\&]
			\Rogers[\sigma]{r-1} \ar[d] \ar[r] \&
			\X_{\ham,\sigma} \ar[r,"\iota_{\dots}\sigma"] \ar[d,"\Lie"]\&
			(\Omega^0\to\Omega^1\to\cdots\to\Omega^{r-1}\to \d \Omega^{r-1}) \ar[d,hook] 
			\\
			\Courant[\sigma]{r-1} \ar[r] \ar[d]\&
			\g_{r+1,\sigma}^{~\geq 0} \ar[r,hook] \ar[ur,Rightarrow] \ar[d,"\Phi\big\vert_{\geq 0}"]\& 
			\g_{r+1,\sigma} \ar[d,"\Phi"]
			\\
			\Courant{r} \ar[r] \&
			\g_{r}^{\geq 0} \ar[r,hook]\&
			\g_{r}
		\end{tikzcd}
	\end{displaymath}
\end{frame}
%-----------------------------------------------------------%


%-----------------------------------------------------------%
\begin{frame}
 .
\end{frame}
%-----------------------------------------------------------%














%-----------------------------------------------------------%
\ifstandalone
% https://en.wikibooks.org/wiki/LaTeX/Bibliographies_with_biblatex_and_biber
\begin{frame}[t,allowframebreaks]{Partial Bibliography}
	\nocite{Miti2021}
	\bibliographystyle{alpha}
	\bibliography{bibfile}
\end{frame}
\fi
%-----------------------------------------------------------%



%===========================================================%
\end{document}
%===========================================================%